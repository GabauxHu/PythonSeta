\documentclass[12pt]{article}
\usepackage[magyar]{babel}
%% \documentstyle[12pt, babel, magyar]{article}

\usepackage{makeidx}

\usepackage{graphicx}
\usepackage{float}

\usepackage[magyar]{babel}

\usepackage{fancyvrb}

\usepackage{fancyhdr}
\pagestyle{fancy}

\fancyhead[R]{\today}
\fancyfoot[C]{\thepage}

%% \renewcommand{\headrulewidth}{0.4pt}
%% \renewcommand{\footrulewidth}{0.4pt}


\newcommand{\konnyu}{$\diamondsuit$}
\newcommand{\kozepes}{$\diamondsuit\diamondsuit$}
\newcommand{\nehez}{$\diamondsuit\diamondsuit\diamondsuit$}

%% szamlalok

\newcounter{minta}[section]
\newcommand{\incminta}{\stepcounter{minta}}
%% \newcommand{\theminta}{\arabic{minta}}
\newcommand{\pelda}{\stepcounter{minta} \theminta. p\'elda \vspace{5mm}}

\newcounter{feladat}
\newcommand{\theproblem}[1]{\stepcounter{feladat}  \noindent{\bf \arabic{feladat}. feladat:} {\it #1}}

\newcounter{megold}
\renewcommand{\themegold}{\stepcounter{megold}  \noindent{\bf \arabic{megold}. megold\'as:}}

\setlength\paperheight {297mm}
\setlength\paperwidth  {210mm} 

\setlength{\baselineskip}{8mm}
\setlength{\parskip}{+3pt}
\setlength{\topmargin}{-2cm}
\setlength{\textheight}{23cm}
%% \setlength{\textwidth}{15cm}
\setlength{\textwidth}{14cm}
\setlength{\hoffset}{-1.5cm}

\title{S\'eta a Python-al \\ II}

%% \author{Urb\'an G\'abor}

\makeindex

\begin{document}

\maketitle

\thispagestyle{fancy}

\begin{abstract}
Aki az el\H{o}z\H{o} f\"uzetet \'atolvasta, \'es a feladatokat meg is oldotta, a Python 
programoz\'asi nyelv alapjaival m\'ar tiszt\'aban kell legyen. Ebben a folytat\'asban ez\'ert 
m\'ar nem a nyelv elemeire fogunk els\H{o}sorban figyelni, hanem a kieg\'esz\'{\i}t\H{o} 
eszk\"oz\"okre (modulokra). Igyekezni fogunk arra is, hogy mag\'ar\'ol a programoz\'asr\'ol 
is b\H{o}v\'{\i}ts\"uk az olvas\'ok ismereteit. A f\"uggel\'ekben m\'elyebb t\'emak\"or\"oket
is \'erint\"unk, valamint a vil\'agh\'al\'on megtal\'alhat\'o v\'abbi dokumentumokat is 
javaslunk az \'erdekl\H{o}d\H{o}knek.
\end{abstract}

\tableofcontents

\section{Bevezet\H{o} gondolatok}

A kor\'abbi s\'et\'ank alapj\'an az olvas\'oink m\'ar meg\'{\i}rtak egy-k\'et 
egyszer\H{u}bb programot, pontosabban szkriptet. Az ott bemtutatott h\'arom alapfeladat 
megold\'as\'anak ismerete tulajdonk\'eppen elegend\H{o} nagyj\'ab\'ol egy hivat\'asos 
programoz\'onak is. Term\'eszetesen ezen a t\'eren is van m\'eg az\'ert gyakorolni val\'o, 
de ez csak id\H{o} k\'erd\'ese. 

A programoz\'oi gondolkod\'ashoz azonban m\'eg sz\"uks\'eg van a nyelv b\H{o}vebb 
lehet\H{o}s\'egeinek megismer\'es\'ere, illetve a programok k\'odol\'as\'aval 
kapcsolatos technikai fog\'asok elsaj\'at\'{\i}t\'as\'ara. Ezek a m\'odszerek a 
legt\"obb korszer\H{u} programoz\'asi nyelvben a rendelkez\'esre \'allnak, ebben 
a le\'{\i}r\'asban az idev\'ag\'o szabv\'anyos Python modulokat \'es technik\'akat 
fogjuk bemutatni. A bevethet\H{o} eszk\"ozt\'ar el\'egg\'e b\H{o}s\'eges, ebb\H{o}l 
mi f\'elig-meddig \"onk\'enyesen v\'alogattunk. Azt az\'ert engedtess\'ek nek\"unkl 
megjegyezni, hogy az itt megmutatott m\'odszerek \'es modulok az\'ert nem olyan nagy 
programoz\'oi gyakorlattal is haszn\'alhat\'ok. A v\'alogat\'asn\'al fontos szempont 
volt az, hogy olyan modulok szerepeljenek, amelyek mag\'at a programoz\'ast seg\'{\i}tik, 
szerencs\'ere ilyen sok van. Itt t\'enyleg csak n\'eh\'any v\'azlatos bemutat\'as\'ara 
volt lehet\H{o}s\'eg. A k\'{\i}v\'ancsi olvas\'oink a f\"uggel\'ekben b\H{o}vebb 
le\'{\i}r\'asokat, az Intereneten el\'erhet\H{o} (sajnos t\"obbynire angol nyelv\H{u}) 
dokumentumokat, \'as a m\'ar megszokott m\'odon az (ism\'et) kihagyott, fontosabb 
t\'emak\"or\"ok list\'aj\'at tal\'alj\'ak.

Az itt szerepl\H{o} fejezetek egym\'ast\'ol majdnem teljesen f\"uggetlenek, ez\'ert 
ak\'ar tetsz\H{o}leges sorrendben is olvashat\'ok. Az esetleges \"osszef\"ugg\'eseket 
\'es bels\H{o} hivatkoz\'asokat minden esetben meg fogjuk eml\'{\i}teni.

\section{B\H{o}v\'{\i}tett Python formai elemek}

Maga a Python nyelv is v\'altozik, fejl\H{o}dik, ahogy a felhaszn\'al\'ok megfogalmazz\'ak 
az ig\'enyeiket. (Ez egy\'ebk\'ent \'altal\'aban jellem\H{o} a nyitott fejleszt\'es\H{u} 
programokra.) Ennek megfelel\H{o}en id\H{o}Nk\'ent \'uj elemek is megejelnhetnek, de ezzel 
aazonban fentart\'asokkal kell b\'anni. Nem jelent el\H{o}nyt, ha mindig azonnal alkalmazzuk 
a frissen megjelen\H{o} \'uj\'{\i}t\'asokat, hiszen ilyenkor c\'elszer\H{u} megv\'arni a 
visszajelz\'eseket. Most k\'et olyan viszonylag \'uj nyelvi elemet fogunk bemutatni, amely a 
tapasztalatok alapj\'an el\'egg\'e stabilan alkalmazhat\'ok az internetes f\'orumok szerint.

\subsection{Dekor\'atorok}
\index{dekor\'ator}

A dekor\'atorok (egyel\H{o}re ez csak az angol megfogalmaz\'as egyszer\H{u} ford\'{\i}t\'asa) 
f\"uggv\'enyek defin\'{\i}ci\'oja alkalm\'aval haszn\'alhat\'ok, amellyel annak \'ertelmez\'es\'et 
pontos\'{\i}tjuk, illetve m\'odos\'{\i}tjuk. A Python programoz\'asi nyelvben van ugyanerre m\'as 
szintaktikai lehet\H{o}s\'eg is, ahogyan az a k\"ovetkez\H{o} p\'eld\'an is l\'atszik: 

\begin{Verbatim}[fontsize=\small]
def f(...):
    ...

f = staticmethod(f)

    ...
    ...

@staticmethod
def f(...):
    ...
\end{Verbatim}

\noindent A fenti minta ugyanarra tartalmazza mind a k\'et lehet\H{o}s\'eget, \'es megmutatja a 
dekor\'atorok haszn\'alat\'at. Mind a k\'et r\'esz az {\tt f} f\"uggv\'enyt \'ugy hat\'arozza meg, 
hogy az els\H{o} argumentuma nem lehet alap\'ertelmezett. (Ezt jelenti ugyanis a {\tt staticmethod})
kulcssz\'o. Dekor\'atorokra van m\'eg p\'elda, tal\'an a leghasznosabbat a \pageref{skiptest}. oldalon 
tal\'alhatjuk.

\subsection{K\"ornyezet-kezel\H{o}k}
\index{with}

Ahogyan a fentebb mutatott dekor\'atorok is els\H{o}rban a k\'enyelmes k\'odol\'asban jelentenek 
\'ujdons\'agot, ugyanez \'erv\'enyes most ter\'{\i}t\'ekre nyelvi elemekre is. (M\'eg ehhez sem 
tal\'altunk bevett magyar fogalmat, ez szint\'en csak az angol kifeje\'es ford\'{\i}t\'asa.) Sok 
esetben van arra sz\"uks\'eg bizonyos felt\'etelek ellen\H{o}rz\'es\'ere a k\'od tov\'abbi sikeres 
fut\'as\'ahoz, erre a kor\'abban megismer kiv\'etelek \'es az {\tt if -- else } utas\'{\i}t\'as 
t\"ok\'eletesen alkalmas. Az \'uj {\tt with} elem ezt ugyanakkor leegyszer\H{u}s\'{\i}ti:

\begin{Verbatim}[fontsize=\small]
with open('adat.txt','r') as text:
    ...
\end{Verbatim}

\noindent Ebben a k\'odr\'eszletben kicsit el\H{o}reszaladunk, mert tal\'an a legk\'ezenfekv\H{o}bben 
mutatja meg a m\H{u}k\"od\'es\'et. A sz\'am\'{\i}t\'og\'ep v\'egrehajtja a {\tt with} ut\'an 
k\"ozvetlen\"ul \'all\'o utas\'{\i}t\'ast, \'es csak abban az esetben ugrik be blokkba, ha az 
sikeres volt. Ebben a blokkban \'atmenetileg \'elni fog a {\tt text} nev\H{u} v\'altoz\'o. 
(Az {\tt open} h\'{\i}v\'as r\'eszleteir\H{o}l a k\"ovetkez\H{o} fejezetben fogunk t\"obbet \'{\i}rni.)

\subsection{Gener\'atorok iter\'atorok \'es {\tt yield}}
\index{gener\'ator}
\index{yield}
\index{iter\'ator}

Ebben a r\'eszben ugyan meg fogunk ismerni egy-k\'et \'uj Python nyelvi elemet, de maga a kindul\'o 
\"otlet egyr\'eszt ismer\H{o}segyszer\H{u}, m\'asr\'eszt pedig m\'ar ismer\H{o}s kell legyen azoknak 
az olvas\'oknak, akik az els\H{o} f\"uzet\"unket \'atn\'ezt\'ek. Vegy\"unk most onnan egy j\'ol 
\'erteht\H{o} p\'eld\'at:

\incminta
\begin{Verbatim}[fontsize=\small]
# for1.py

lista = ['Isti', 'Peti', 'Joci'] 

for elem in lista:
   print( ' > ' + elem)

print('Vege')
\end{Verbatim}
\theminta. p\'elda

\noindent Most m\'ar nem is kell ezt lefuttatni, tudjuk hogy az {\tt elem} v\'altoz\'o sorra 
egym\'as ut\'an felveszi a lista elemeinek \'ert\'eket. Hasonl\'oan vil\'agos, a:

\begin{Verbatim}[fontsize=\small]
for i in range(12):
   .... 
\end{Verbatim}

\noindent ahol az {\tt i} \'ert\'eke sorra felveszi 0-t\'ol 11-ig a sz\'amokat, {\bf mintha} azt 
is egy list\'ab\'ol venn\'e. (Val\'oj\'ab\'ol a bels\H{o} m\H{u}k\"od\'es szempontj\'ab\'ol 
t\'enyleg egy lista.) A k\'et {\tt for} ciklusban az a k\"oz\"os, hogy valamilyen sorrendben 
v\'egigmegy egy halmaz minden elem\'en, programoz\'oi \'es matematikai szakkifejez\'essel 
{\sl iter\'al}. Vagyis a {\tt for} utas\'{\i}t\'as egy {\sl iter\'ator.}

A Python egyik hasznos nyelvi eleme, hogy iter\'atorokat saj\'at magunk is tudunk k\'esz\'{\i}teni, 
pontosabban l\'etezik egy \'altal\'anos iter\'ator t\'{\i}pus, amelyet fel tudunk haszn\'alni. 
(Tulajdonk\'eppen ezt haszn\'alja a {\tt for} ciklus is, amikor egy lista elemein megy v\'egig.) 
Terjedelmi megfontol\'asok miatt ennek a r\'eszleteibe most nem fogunk belemenni. A tov\'abbi 
ismerked\'es\"unkat a {\tt yield} utas\'{\i}t\'assal fogjuk folytatni. Erre egy p\'elda:

\begin{Verbatim}[fontsize=\small]
def echo(value=None):
    print(" Start, ha 'next()' eloszor lesz meghivva.")
    try:
        while True:
            try:
                value = (yield value)
            except Exception as e:
                value = e
    finally:
                print("Tisztitas, ha 'close()' meg lett hivva.")

print('-- 1 --\n')
generator = echo(1)
print('-- 2 --\n')
print(next(generator))
print('-- 3 --\n')
print(next(generator))
print('-- 4 --\n')
print(generator.send(2))
print('-- 5 --\n')
generator.throw(TypeError, "spam")
print('-- 6 --\n')
generator.close()
\end{Verbatim}

\noindent A fenti szkript lefuttat\'asa sor\'an a k\"ovetkez\H{o} eredm\'enyt kapjuk:

\begin{Verbatim}[fontsize=\small]
-- 1 --

-- 2 --

 Start, ha 'next()' eloszor lesz meghivva.
1
-- 3 --

None
-- 4 --

2
-- 5 --

-- 6 --

Tisztitas, ha 'close()' meg lett hivva.
\end{Verbatim}

\noindent Ehhez a kimenethez egy-k\'et sz\'ot az\'ert sz\'olni. A legels\H{o} f\"uggv\'enyh\'{\i}v\'askor 
nem t\"ort\'enik semmi, ha f\"uggv\'eny eredm\'eny\'et tartalmaz\'o v\'alto\'ot ki\'{\i}rnn\'ank a 
k\'eperny\H{o}re, akkor egy t\'arhely hivatkoy\'ast l\'atn\'ank. A {\tt next} f\"uggv\'enyh\'{\i}v\'as 
viszont elind\'{\i}tja a gener\'al\'ast, amint az l\'atszik. Teh\'at a {\tt yield} utas\'{\i}t\'assal 
egy gener\'atort hoztunk l\'etre. A Python keretrendszer automatikusan hozz\'aillesztett egy {\tt send} 
\'es egy {\tt throw} met\'odust, az els\H{o} a gner\'atoron kereszt\"ul \'ert\'eket ad vissza, a m\'asodik 
pedig egy kiv\'etelt dob. Ez ut\'obbi most nem t\"ort\'ent meg, amikor a k\'odot egyben, szkriptk\'ent futtattuk 
le. Ha ezzel szemben egy Python parancs\'ertelmez\H{o}be soronk\'ent m\'asoljuk be a k\'odot, \'es \'{\i}gy 
``k\'ezzel'' hajtjuk v\'egre, akkor \'erdekes m\'odon a hiba\"uzenet megjelenik. A gener\'atort a v\'eg\'en 
a {\tt close} met\'odussal z\'arjuk le.

\section{\'Allom\'anyok kezel\'ese}
\label{file}

\subsection{\'Alltal\'anos megjegyz\'esek}

Miel\H{o}tt bele\'asn\'ank magunkat a az \'allom\'anykezel\'es r\'eszleteibe, 
fontosnak tartunk n\'eh\'any \'altal\'anos megjegyz\'est. A Python ezeket az 
adatsorokat mindig objektumk\'ent kezeli, de ezek a konstruktorral \index{open}
k\"ozvetlen\"ul nem p\'el\-d\'a\-nyos\'{\i}that\'ok. Erre minden esetben a szabv\'anyos 

\incminta
\begin{Verbatim}[fontsize=\small]
   allomany = open( <nev>, <mod>)
\end{Verbatim}
\theminta. p\'elda

\noindent f\"uggv\'enyh\'{\i}v\'as \'all a rendelkez\'es\"unkre. A f\"uggv\'eny mind 
a k\'et param\'etere karakterl\'anc (string), \'es a jelent\'es\"uk a r\"oviden a 
k\"ovetkez\H{o}:

\begin{itemize}
   \item{{\tt <nev>} a megnyitand\'o \'allom\'any neve}
   \item{{\tt <mmod>} a haszn\'alat m\'odja}   
\end{itemize} 

\noindent Ezekr\H{o}l, f\H{o}leg a m\'asodikr\'ol majd b\H{o}vebben is fogunk ebben a 
f\"uzetben olvasni.

Gondoljuk, azt minden olvas\'onk tudja, hogy az \'allom\'anyok a sz\'am\'{\i}t\'og\'ep 
lemezegys\'eg\'en j\"onnek l\'etre, illetve onnan lesznek besolvasva. Err\H{o}l a 
k\'erd\'esr\H{o}l is lesz b\H{o}vebben sz\'o, itt el\'egedj\"unk meg azzal az 
egyszer\H{u}s\'{\i}t\'essel, hogy minden esetben abban a k\"onyvt\'arban (mapp\'aban) 
lesznek az \'allom\'anyaink, ahol a Python szkript van, \'es ahol futtatjuk. A tov\'abbiakban 
ezt term\'eszetesen majd \'arnyaljuk. Most csak fut\'olag eml\'{\i}tj\"uk meg, hogy mindez 
nagy m\'ert\'ekben f\"ugg att\'ol is (a kor\'abbiakkal ellent\'etben), hogy milyen g\'epen 
dolgozunk, vagyis az {\sl oper\'aci\'os rendszert\H{o}}l, \index{oper\'aci\'os rendszer} 
illetve az az \'altal haszn\'alt {\sl f\'ajlrendszert\H{o}l}. \index{f\'ajlrendszer}

Miel\H{o}tt gyorsan menn\'enk tov\'abb, egyetlen dolgot m\'eg el kell \'arulnunk az 
Olvas\'oinknak.A m\'ar ismertett f\"uggv\'ennyel meg tudunk nyitni egy \'allom\'anyt, 
de a programban azt id\H{o}nk\'ent le is kell z\'arni. Erre val\'o a \index{close}

\incminta
\begin{Verbatim}[fontsize=\small]
   allomany.close()
\end{Verbatim}
\theminta. p\'elda

\noindent h\'{i}v\'as, ami a lez\'ar\'as mellett meg is semmis\'{\i}ti az \'allom\'any-objektumot. 
Ha tetszik, ez\'ert tekinthetj\"uk ak\'ar destruktornak is, de p\'eld\'anyos\'{\i}tani sem a 
konstruktorral p\'eld\'anyos\'{\i}tottuk. Beolvas\'askor ez nem annyira fontos, de amikor 
adatokat \'{\i}runk ki, akkor nem szabad r\'ola elfeledkezni, mert kellemetlen meglepet\'esekkel 
j\'arhat. Err\H{o}l a k\"ovetkez\H{o} r\'eszen m\'eg \'{\i}runk. 

\subsection{\'Allom\'anyok \'{\i}r\'asa}

Kezdj\"uk is mindj\'art az egyszer\H{u}bb t\'em\'aval, az \'allom\'anyok \'{\i}r\'as\'aval. 
Az elej\'en azonban m\'aris van egy k\'erd\'es, amit el kell d\"onten\"unk, vagyis \'uj 
adatsort akarunk-e l\'etrehozni, vagy egy kor\'abbit kieg\'esz\'{\i}teni. Mind a k\'et 
esetben a megnyit\'askor a m\'od param\'etert kell megfelel\H{o}en be\'all\'{\i}tni:

\begin{itemize}
   \item{{\tt "w"}, ha az \'allomanyt \'{\i}rni akarjuk}
   \item{{\tt "a"}, ha az adatokhoz hozz\'af\H{u}z\"unk.}   
\end{itemize} 

\noindent A megadott n\'even  neml\'etez\H{o} \'allom\'any eset\'en mind a kett\H{o} egyszer\H{u}en 
l\'et\-re\-hoz\-za azt \"uresen, \'es m\'aris mehetnek bele az adatok. Gy\"okeresen m\'as a helyzet 
azonban akkor, ha ilyen nev\H{u} f\'ajl m\'ar l\'etezik. Ekkor {\tt "w"} haszn\'alata eset\'en a 
kor\'abbi tartalma teljesen {\bf t\"orl\H{o}dik}, hiszen a param\'eter azt jelezte, \'uj adatsor fog 
k\'esz\"ulni. Ha viszont az \'atadott param\'eter {\tt "a"}, akkor az adott \'allom\'any kor\'abbi 
tartalma megmarad, hiszen hozz\'a akarunk f\H{u}zni.  Ha a megnyit\'as valamilyen okn\'al fogva 
m\'egse siker\"ul, akkor a Python a szok\'asos m\'odon kiv\'etelt dob, ezzel jelzi a hib\'at. 
L\'assunk az\'ert k\'et egyszer\H{u} p\'eld\'at:

\incminta
\begin{Verbatim}[fontsize=\small]
adat = open("teszt.txt", "w")
sor = open("proba.3", "a")
\end{Verbatim}
\theminta. p\'elda

\noindent Ha a megnyit\'as \'{\i}gy, vagy \'ugy sikeres volt, akkor m\'ar k\"uldhetj\"uk az \'allom\'anyba 
az adatokat, p\'eld\'aul:

\incminta
\begin{Verbatim}[fontsize=\small]
   adat.write("Ez egy sor")
\end{Verbatim}
\theminta. p\'elda

\noindent Az utas\'{\i}t\'as hat\'as\'ara az \'atadott string ki\'{\i}r\'odik az \'allom\'anyba, de 
sz\"ovegr\H{o}l l\'ev\'en sz\'o, fontos eml\'ekezn\"unk arra, hogy a soremel\'es nem ad\'odik hozz\'a, 
az a programoz\'o feladata. Mondjuk: 

\incminta
\begin{Verbatim}[fontsize=\small]
   adat.write("Ez egy sor\n")
\end{Verbatim}
\theminta. p\'elda

T\'etelezz\"uk fel azt az esetet, amikor egyszerre t\"obb sort szeretn\'enk az \'allom\'anyba kiiratni. 
Szerencs\'ere nem kell nek\"unk ilyenkor ciklust \'{\i}rni, a Python programoz\'asi nyelv ad egy eleg\'ans 
megold\'ast:

\incminta
\begin{Verbatim}[fontsize=\small]
   adat.writelines(sorok)
\end{Verbatim}
\theminta. p\'elda

\noindent Term\'eszetesen a soremel\'esekr\H{o}l itt is a nek\"nk kell k\"ul\"on gondoskodnunk. Gyakorlott 
programoz\'ok m\'ar a list\'at is ennek megfelel\H{o}en rakj\'ak \"ossze.

Az \'allom\'anyok \'{\i}r\'asakor a lez\'ar\'asnak k\"ul\"onleges szerepe van. Az adatok ugyanis nem 
azonnal ker\"ulnek ki a lemezre, hanem el\H{o}sz\"or egy \'atmeneti t\'arol\'oba (\'ugynevezett 
{\sl pufferbe} t\"olt\H{o}dnek be, \index{puffer} \'es csak akkor \'{\i}r\'odnak ki, ha a puffer megtelik. 
A {\tt close} \index{close} viszont azonnal elkezni a puffer lement\'es\'et.Ez\'ert az a legbiztons\'agosabb, 
ha minden {\tt write} \'es {\tt writelines} ut\'an az \'allom\'anyt lez\'arjuk, majd ut\'ana \'ujranyitjuk.

\subsection{\'Allom\'anyok beolvas\'asa}

Ha fel akarunk dolgozni egy \'allom\'anyt a megnyit\'asi m\'od:

\begin{itemize}
   \item{{\tt "r"}, ebben az esetben}   
\end{itemize} 

\noindent Teh\'at a teljes megnyit\'as:

\incminta
\begin{Verbatim}[fontsize=\small]
adat = open("teszt.txt", "r")
\end{Verbatim}
\theminta. p\'elda

\noindent Mivel most csak egyetlen m\'od \'all a rendelkez\'es\"unkre, el is kezdhetj\"uk a beolvas\'ast:

\incminta
\begin{Verbatim}[fontsize=\small]
sor = adat.readline()
\end{Verbatim}
\theminta. p\'elda

\noindent Els\H{o} beolvas\'askor az elej\'et\H{o}l szedi be lemezr\H{o}l az adatokat a program, ut\'ana 
pedig a f\'ajl-mutat\'o pillanatnyi helyzet\'et\'et\H{o}l. Az esetek t\"obbs\'eg\'eben nem haszn\'alunk 
semmif\'ele param\'etert, de err\H{o}l kicsit k\'es\H{o}bb m\'eg lesz sz\'o.  

Van a Pythonban arra is lehet\H{o}s\'eg, hogy egyszerre t\"obb sort olvassunk be az \'allom\'anyb\'ol 
(az \"osszeset, vagy a marad\'ekot), erre szolg\'al a m\'asik utas\'{\i}t\'as:

\incminta
\begin{Verbatim}[fontsize=\small]
sorok = adat.readlines()
\end{Verbatim}
\theminta. p\'elda

\noindent A f\"uggv\'enyben itt sem haszn\'altunk egyel\H{o}re param\'etert. Fel kell h\'{\i}vni mindenkinek 
a figyelm\'et arra, hogy esetleg nagyon sok adatot fogunk ilyenkor beolvasni. 

TODO SOREMEL?????

Eddig sorokr\'ol besz\'elt\"unk, de az \'allom\'anyok egy nagy r\'esze nem sz\"oveges, illetve nem 
tagol\'odik sorokra. Sz\"uks\'eg van teh\'at egy olyan f\"uggv\'enyre is, amelynek meg tudjuk mondani, 
hogy az \'allom\'anyb\'el mennyit akarunk beolvasni:

\incminta
\begin{Verbatim}[fontsize=\small]
import os
adatok = os.read(adat,<darab>)
\end{Verbatim}
\theminta. p\'elda

\noindent Ahogyan a p\'eld\'abol l\'atjuk, ehhez sz\"uks\'eg volt az {\tt os} modul import\'al\'as\'ara 
is. Ez azokat a f\"uggv\'enyeket tartalmazza, amelyek a lemezegys\'eggel kapcsolatos m\H{u}veleteket 
val\'os\'{\i}tja meg, k\'es\H{o}bb m\'eg b\H{o}vebben is ismerets\'eget k\"ot\"unk vele.

A kor\'abbi \'{\i}g\'erethez h\'{\i}ven t\'erj\"unk most vissza a beolvas\'o met\'odusok param\'etereihez. 
L\'assuk most az els\H{o} p\'eld\'at:

\incminta
\begin{Verbatim}[fontsize=\small]
sor = adat.readline(<darab>)
\end{Verbatim}
\theminta. p\'elda

\noindent Az utas\'{\i}t\'as a kor\'abbihoz hasonl\'oan beolvas egy sort, ha a karakterek sz\'ama nem nagyob, 
mint a megadott darabsz\'am. Abban az esetben, ha a sor t\'eny\-le\-ge\-sen hosszabb, akkor legfelejebb a megadott 
sz\'am\'u karaktert olvassa be. Ilyenkor a sor tov\'abbi r\'esz\'et ism\'etelt h\'{\i}v\'assal tudjuk beolvasni.
Hasonl\'o a helyzet a 

\begin{Verbatim}[fontsize=\small]
sorok = adat.readlines(<darab>)
\end{Verbatim}

\noindent Ilyenkor a megadott darabsz\'am a sorok \"ooszes hossz\'at jelenti.

TODO utols\'o sor??

\subsection{Tov\'abbi megjegyz\'esek az \'allom\'anyokhoz}

Az el\H{o}z\H{o} k\'et r\'eszben a legfontosabb dolgokat megimsert\"uk, de ez sajnos nem jelenti azt, 
hogy eleget tudunk nyakatekertebb feladatok megold\'as\'ahoz. Mindenekel\H{o}tt l\'enyeges annak 
be\'all\'{\i}\'asa, hogy az \'allom\'anyokat melyik k\"onyvt\'arba (mapp\'aba) hozza l\'etre, 
illetve melyik k\"onyvt\'arban (mapp\'aban) keresse olvas\'askor a sz\'am\'{\i}t\'og\'ep. Tov\'abb\'a 
id\H{o}nk\'ent le kell ellen\H{o}rizni, hogy a feldolgozand\'o f\'ajl, illetve a haszn\'aland\'o mappa 
t\'enylegesen l\'etezik-e. Mindehhez egy kicsit meg kell ismern\"unk az \'allom\'anyokat t\'arol\'o 
\index{f\'ajlrendszer} {\sl f\'ajlrendszer} kezel\'es\'enek r\'eszleteit.\footnote{\'Erdekess\'egk\'ent 
jegyezz\"uk, meg, hogy egyes oper\'aci\'os rendszerek, p\'eld\'aul a Linux t\"obbf\'ele f\'ajlrendszert 
is tud kezelni.} Els\H{o}sorban olvas\'as eset\'en van arra lehet\H{o}s\'eg, hogy ne csak sorban dolgozzuk 
fel az adatokat, hanem tudjunk a f\'ajokban el\H{o}re \'es h\'arta mozogni. A megnyit\'asi m'odokr\'ol is 
kell m\'eg b\H{o}vebben mondanunk valamit, hiszen ezen a t\'eren is sok lehet\H{o}s\'eget biztos\'{\i}t 
sz\'amunkra a Python programoz\'asi nyelv.

\subsubsection{Kapcsolat a f\'ajlrendszerrel}

Minden f\'ajlrendszer (\'es minden oper\'aci\'os rendszer) m\'as \'es m\'as, de szeren\-cs\'enk\-re ezt 
a programoz\'asi nyelv el\H{o}l\"unk eltakarja, elegend\H{o} csak a m\'ar emlegetett {\tt os} modult 
import\'alni. Ez lehet\H{o}v\'e teszi, hogy a programot a fejleszt\'es befejez\'ese ut\'an egy m\'asik 
sz\'am\'{\i}t\'og\'epen haszn\'aljuk, ha csak a szabv\'anyos f\"uggv\'enyeket alkalmazzuk, akkor elvileg 
nem \'erhet minket meglepet\'es. Az alap\'ertelme\'es szerint a munkak\"onyvt\'ar az, ahol a szkriptet 
elind\'{\i}tuttuk, (\'altal\'aban, ahol tal\'alhat\'o). Err\H{o}l viszont meg is tudunk bizonyosodni:

\begin{Verbatim}[fontsize=\small]
import os
print(os.getcwd())
\end{Verbatim}

\noindent Ennek a k\'odnak hat\'as\'ara val\'oban az aa k\"onyvt\'ar \'{\i}r\'odik ki, ahol ezt 
lefuttattuk. Ha kipr\'ob\'aljuk, akkor l\'atjuk, hogy ez megadja a teljes el\'er\'esi \'utvonolat. 
A k\"onyvt\'ar v\'alt\'asra a {\tt chdir} f\"uggv\'eny haszn\'aland\'o:

\begin{Verbatim}[fontsize=\small]
import os
print(os.getcwd())
os.chdir('test')
print(os.getcwd())
\end{Verbatim}

\noindent A r\"ovid szkript lefuttat\'asa ut\'an l\'atjuk, hogy a mappa\'alt\'as t\'enylegesen 
meg is t\"ort\'ent. Gondoljuk, nem kell arra felh\'{\i}vni a figyelmet, ha a {\tt 'test'} 
k\"onyvt\'ar nem l\'etezik, akkor a Python kiv\'etelt fog dobni. K\"onyvt\'arat is tudunk 
l\'etrehozni az {\tt mkdir()} f\"uggv\'ennyel:

\begin{Verbatim}[fontsize=\small]
import os
os.mkdir('uj_mappa')
\end{Verbatim}

\noindent A pr\'obafuttat\'as sor\'an l\'eltrej\"on az \'uj k\"onyvt\'ar, kiv\'eve akkor, ha 
m\'ar l\'etezik. Ez ut\'obbi esetben kiv\'etelt fog a rendszer dobni.

M\'ar eml\'{\i}tett\"uk, hogy a most bemutatott f\"uggv\'enyek hib\'ara tudnank futni, ennek 
kezel\'es\'ere a Python nyelvben k\'et lehet\H{o}s\'eg \'all a rendelkez\'es\"unkre. Az 
egyik, a legt\"obb Python felhaszn\'al\'o ezt szokta haszn\'alni, hogy a k\'erd\'eses 
utas\'{\i}t\'ast bez\'arjuk m\'ar ismert m\'odon egy {\tt try ... except} szerekezetbe, 
\'es az \'{\i}gy elkapott {\tt PermissionError} kiv\'etelt lekezelj\"uk. A m\'asik megold\'as, 
hogy a "k\'enyes" m\H{u}veletek el\H{o}tt leellen\H{o}rizz\"uk, hogy a megfelelel\H{o} elem 
l\'etezik-e a f\'ajlerndszerben, erre val\'o az {\tt os.access()} f\"uggv\'eny:

\begin{Verbatim}[fontsize=\small]
import os
teszt = os.access("myfile", os.R_OK)
\end{Verbatim}

\noindent A h\'{\i}v\'as eredm\'enyek\'ent logikai igaz ({\tt True}), vagy hamis ({\tt False}) 
\'ert\'eket ad vissza. Az els\H{o} param\'eter az ellen\H{o}rzend\H{o} elem neve (pontosabban 
a teljes el\'er\'esi \'utja), a m\'asodik pedig az olvashat\'os\'agot jelenti. Ez a m\'od 
\'ert\'ek szint\'an az import\'alt modulban vand defini\'alva. Ha csak az elem l\'etez\'es\'et 
akarjuk tesztelni, akkor a m\'od param\'eter os.F\_OK. T\"obb \'ert\'ert\'eket mag\'aban 
a modulban tal\'alhatunk.

\subsubsection{\'Allom\'anyok \'{\i}r\'as\'ar\'ol \'es olvas\'as\'ar\'ol b\H{o}vebben}

A kor\'abbiakban megmutattuk az \'allom\'anyok megnyit\'asakor haszn\'alt legfontosabb m\'odokat, 
de a Python programoz\'asi nyelven ann\'aol azonban t\"obb lehet\H{o}s\'eg van. Az egyik leg\'erdekesebb, 
hogy meg lehet mondani, milyen jelleg\H{u} az \'allom\'any. Ha tudjuk el\H{o}re, hogy nem sz\"oveges 
adatokat tartalmaz, akkor {\sl bin\'aris} adatk\'ent kell megnyitni, p\'eld\'ul ilyen lehet egy 
k\'epf\'ajl, vagy hangsor. Erre az al\'abbi m\'odok \'allnak a rendelkez\'es\"unlre:

\begin{center}
   \begin{tabular}{cl}
   {\bf M\'od} & {\bf Jelent\'es} \\
   {\tt rb} & Az \'allom\'anyt olvas\'asra nyitja meg bin\'arisan. \\
   {\tt wb} & Az \'allom\'anyt \'{\i}r\'asra nyitja meg bin\'arisan. \\
   {\tt ab} & Az \'allom\'anyt hozz\'af\H{u}z\'esre nyitja meg bin\'arisan. \\
   \end {tabular}
\end{center}

Fontos azonban megjegyezni, hogy ezt nem minden oper\'aci\'os rendszer eset\'en kell elkelmazni, 
ugyanis van olyan, amely minden esetben a bin\'aris tekinti alap\'ertelmez\'esnek. 

Elk\'epzelhet\H{o} a programoz\'oi gyakorlatban, hogy a megynit\'as sor\'an egyszerre van 
sz\"uks\'eg\"nk arra, hogy valamilyen m\'odon \'{\i}rni \'es olvasni is tudjuk az adatsort. 
Erre kombin\'alt megnyit\'asi m\'odok vannak:

\begin{center}
   \begin{tabular}{cl}
   {\bf M\'od} & {\bf Jelent\'es} \\
   {\tt r+} & Az \'allom\'anyt olvas\'asra \'es \'{\i}r\'asra nyitja meg. \\
   {\tt w+} & Az \'allom\'anyt \'{\i}r\'asra \'es olvas\'asra nyitja meg. \\
   {\tt a+} & Az \'allom\'anyt hozz\'af\H{u}z\'esre \'es olvas\'asra nyitja meg. \\
   \end {tabular}
\end{center}

\noindent Hogy az olvas\'oinkat a lehet\H{o} legteljesebb m\'ert\'ekben megzavarjuk, megeml\'{\i}tj\"uk, 
a most le\'{\i}rt m\'oddal szabadon keverhet\H{o} a bin\'aris megnyit\'as.

\subsubsection{Mozg\'as az \'allom\'anyokban}

N\'eha el\H{o} szokott fordulni, hogy egy-egy f\'ajl  eset\'eben nem egym\'as ut\'an kell az adatokat 
feldolgozni, hanem valamilyen m\'odon \'at kell r\'eszleteket ugrani. Ahogyan m\'as a {\tt read} \'es 
{\tt readline} met\'odusok eset\'eben is eml\'{\i}tett\"uk, az minden esetben az \'allom\'anymutat\'o 
pillanatnyi helyzet\'et\H{o} olvassa be az adatokat. Felmer\"ul teh\'at egy\'ertelm\H{u}en a k\'erd\'es, 
van-e lehet\H{o}s\'eg a mutat\'o \'ert\'ek\'et lek\'erdezni, illetve megv\'altoztatni. A Python 
f\'ajlkezel\H{o} met\'odusai mind a kett\H{o}t megoldj\'ak a sz\'amunkra. Az els\H{o} {\tt tell}, a 
m\'asodikra pedig a {\tt seek} f\"uggv\'eny alkalmazhat\'o. L\'assuk a gyakorlatban:

\begin{Verbatim}[fontsize=\small]
with open('tell.tst','r') as inp:
   pos = inp.tell()
   print('1-- %d'%pos)
   sor = inp.readline()
   print(sor)
   pos = inp.tell()
   print('2-- %d'%pos)
   inp.seek(24,0)
   pos = inp.tell()
   print('3-- %d'%pos)
   sor = inp.readline()
   print('teszt')
   print(sor)
\end{Verbatim}

\noindent A fenti k\'odot lefuttattuk (a bemeneti adatf\'ajl sz\a'msorokat tartalmaz egym\'as alatt) .Az eredm\'eny:

\begin{Verbatim}[fontsize=\small]
python telltest.py
1-- 0
1234

2-- 6
3-- 24
teszt
432
\end{Verbatim}

\noindent A kimenet \'ertelmez\'ese nem olyan neh\'ez, a megnyit\'as ut\'an az \'allomy\'anymutat\'o az elej\'ere 
\'allt be, majd az els\H{o} sor beolvas\'asa ut\'an a m\'asodik sor elej\'ere. A {\tt seek} f\"uggv\'eny 
megh\'{\i}v\'asa az \'allom\'anymutat\'ot k\"ozvetlen\"ul m\'odos\'{\i}tja, az azt k\"ovet\H{o} kiirat\'as 
tanuls\'aga alapj\'an t\'enyleg a megadott poz\'{\i}ci\'ora \'all\'{\i}tja. M\'elyedj\"unk el kiss\'e ennek a 
f\"uggv\'enynek a m\H{u}k\"od\'es\'ebe. Az els\H{o} argumentum a k\'{\i}v\'ant poz\'{\i}ci\'ot jelenti, de 
ennek helyes \'ertelmez\'ese a m\'asodik param\'etert\H{o}l f\"ugg. Az al\'abbi t\'abl\'azat ebben seg\'{\i}t:

\begin{center}
   \begin{tabular}{cll}
      {\bf M\'asodik param\'eter} & {\bf Els\H{o} \'ert\'ek} & {\bf Jelent\'es} \\
      {\tt 0, SEEK\_SET} & Csak pozit\'{\i}v & Az \'allom\'any elej\'et\H{o}l sz\'am\'{\i}tva \\
      {\tt 1, SEEK\_CUR} & Minden lehet & Az aktu\'asli helyhez k\'epest kell megandi \\
      {\tt 2, SEEK\_END} & Csak negat\'{\i}v & Az \'allom\'any v\'eg\'et\H{o}l sz\'am\'{\i}tva \\      
   \end {tabular}
\end{center}


\section{Dokument\'aci\'o Python m\'odra}
\label{dokument}

Az esetek t\"obbs\'eg\'eben nem elegend\H{o}, ha egy programk\'od j\'ol olvashat\'o, 
a m\H{u}k\"od\'es\'enek meg\'ert\'es\'ehez sz\"uks\'eg van valamif\'ele dokument\'ac\'ora. 
A kor\'abbiak alapj\'an tudunk a Python k\'odba megjegyz\'eseket \'irni, de szerencs\'ere 
a Python programoz\'asi nyelv enn\'el jobb lehet\H{o}s\'eget tud ny\'ujtani. Lehet olyan 
le\'{\i}r\'ast mell\'ekelni a programokhoz \'es mdulokhoz, amelyet ak\'ar egy HTML 
b\"ong\'esz\H{o}vel is tudunk olvasni. Mindez a k\'odb\'ol automatikusan el\H{o} is 
\'all\'{\i}that\'o.

\subsection{A dokument\'aci\'os string}

Minden szkripthez, defini\'alt f\"uggv\'enyhez \'es oszt\'alyhoz lehet dokument\'aci\'ot 
mell\'ekelni a {\sl dokument\'aci\'os stringek} \index{doc string} seg\'{\i}ts\'eg\'evel. Ennek 
szabv\'anyos helye a defini\'al\'o fej ut\'an van. L\'assunk egy egyszer\H{u} p\'eld\'at:

\begin{Verbatim}[fontsize=\small]
def fugvveny(....):
"""
Az lesz a doksi, amit itt van.
"""
   ...
\end{Verbatim}

\noindent A fenti p\'elda tulajdonk\'eppen csak a m\'odszert igyekszik bemutatni, k\"ul\"on\"osebb 
magyar\'azatot nem k\'{\i}v\'anunk hozza\'af\H{u}zni. A Python \'ertelmez\H{o} sz\'am\'ara a 
h\'arom id\'ez\H{o}jel murtatja az ilyen megjegyz\'esek kezdet\'et \'es v\'eg\'et. Vil\'agos, 
hogy ezekre a sorokra nem fog r\'afutni a vez\'erl\'es.

\begin{Verbatim}[fontsize=\small]
todo
\end{Verbatim}

\subsection{A pydoc szkript}
\label{pydoc}

A szkript r\'esze a szabv\'anyos Python rendszernek (neve {\tt python.py}), arra 
lett kifejlesztve, hogy a k\'odba elhelyezett dokument\'aci\'os stringek alapj\'an k\'epes 
legyen valamif\'ele j\'ol olsvashat\'o dokument\'ac\'ot el\H{o}\'all\'{\i}tani. 
Egyr\'eszt egyszer\H{u} sz\"oveges inform\'ac\'ot, m\'asr\'eszt pedig b\"ong\'eszhet\H{o} HTML 
dokumentumot tud gener\'alni. Ezen k\'{\i}v\"ul lehet\H{o}s\'eg van arra, hogy a b\"ong\'esz\H{o}  
sz\'am\'ara ezeket az adatokat ig\'eny eset\'en el\H{o}\'all\'{\i}tsa valamilyen inform\'aci\'os 
szerveren. Pontosabban megfogalmazva egy WEB szerveren futtatva az legy\'artott dokumentumot 
egyb\H{o}l le tudja t\"olteni a t\'avoli g\'epen fut\'o b\"ong\'esz\H{o}re/.

Az aj\'anl\'asok szerint \'erdemes a dokument\'aci\'os stringbe alapvet\H{o} adatokat belevenni. 
L\'assunk erre egy mint\'at:

\begin{Verbatim}[fontsize=\small]
__author__ = "szerzo email"
__date__ = "2020.10.03"
__version__ = "0.1.0" 

\end{Verbatim}

\noindent Ezek az inform\'aci\'ok meg fognak jelenni a legener\'alt dokument\'aci\'oban is. 
(Term\'eszetes a t\'enyleges adatokat kell be\'{\i}rni.) K\'es\H{o}bb m\'as c\'elra is 
azonos\'{\i}tj\'ak a modult, erre majd mutatunk m\'eg p\'eld\'at.

Amikor futtatjuk a {\tt pydoc} szkriptet, akkor minden esetben teljesen \'ertel\-me\-zi azt 
a Python k\'odot, amelyiknek a dokument\'aci\'oj\'at el\H{o} akarjuk vele \'all\'{\i}tani. Ez 
azt is jelenti, hogy legal\'abb is formailag helyesek kell lennie. Magyar\'an egyszer\H{u}en 
arra is haszn\'alhatjuk, hogy kijav\'{\i}tsuk a szintaiktikai hib\'akat. 

Most v\'egre l\'assuk a haszn\'alat\'at, parancsorban kell kiadni:

\begin{Verbatim}[fontsize=\small]
pydoc.py modulom
\end{Verbatim}

\noindent Nem bonyolult, de k\'et megjegyz\'est az\'ert szeretn\'enk hozz\'af\H{u}zni. A 
{\tt pydoc.py} szkriptnek a megfelel\H{o} mapp\'ab\'ol el\'erhet\H{o}nek kell lenni, ezt 
esetleg \'ugy is megoldhatjuk, hogy sim\'an odam\'asoljuk. (MS Windows k\"ornyezetben 
tal\'an \'{\i}gy a legegyszer\H{u}bb.) A m\'asik, hogy a feldolgozand\'o Python szkript 
nev\'et elegend\H{o} a kiterjeszt\'es (t\'{\i}pus) n\'elk\"ul megadni. A fenti kiadott 
parancs hat\'as\'ara az el\H{o}\'all\'{\i}tott dokument\'aci\'o sz\"oveges form\'aban 
megjelenik a k\'eperny\H{o}n. Ha hosszabb, akor egyszerre annyi r\'esz \'{\i}r\'odik 
ki, amennyi egyszerre kif\'er, \'es lapozni kell el\H{o}re.

\begin{Verbatim}[fontsize=\small]
pydoc.py -w modulom
\end{Verbatim}

\noindent Ez ut\'obbi utas\'{\i}t\'as az, amely seg\'{\i}ts\'eg\'evel a b\"ong\'eszhet\H{o} 
HTML \'allom\'anyba \'{\i}r\'odik ki a dokument\'aci\'o, ennek neve {\tt modulom.html}. 
A b\"ong\'esz\H{o}ben a le\'{\i}r\'as szinesen jelenik meg, az alap\'ertelmezett 
sz\'ink\'odokat a szkript \'all\'{\i}tja be.

\section{Modulokr\'ol b\H{o}vebben}

A bevezet\H{o} f\"uzetben m\'ar volt arr\'ol sz\'o, hogy a f\"uggv\'enyek \'es 
oszt\'alyok defin\'{\i}ci\'oj\'at ak\'ar k\"ul\"on Python forr\'as\'allom\'anyba 
is tehetj\"uk, \'es sz\"uks\'eg eset\'en a szkript\"unkbe import\'alhatjuk. Ez 
a m\'odszer nem csak r\"ovidebb\'e teheti a programunkat, hanem abban is seg\'{\i}t, 
hogy olvashat\'obb \'es \'attekinthet\H{o}bb k\'odott tudjunk \'{\i}rni. A 
programoz\'oi hasznos gyakorlat szerint az a legjobb, ha egy modulba a szorosabban 
\"osszetartoz\'o f\"uggv\'enyek \'es oszt\'alyok defin\'{\i}ci\'oi vannak \"osszeszedve. 
Ez az olvashat\'os\'ag mellett abban is seg\'{\i}ts\'eg\"unkre lehet, hogy a modulba 
szervezett k\'odot ak\'ar k\"ul\"on is tesztelhess\"unk. Ezt egy k\"ul\"on fejezetben (\ref{teszt}.) 
fogjuk le\'{\i}rni.

Miel\H{o}tt alaposabban ism\'et bele\'asn\'ank magunkat a modulok haszn\'alat\'anak a 
m\'elys\'egeibe, mindenk\'eppen fontos, hogy l\'assuk, hogyan is kezeli a Python ezeket 
az inform\'aci\'okat. Legyen az egyszer\H{u}s\'eg kedv\'e\'ert a modulunk k\'odja:

\begin{Verbatim}[fontsize=\small]
# Modul A

text = "Haho"

def foo(...)
   pass
\end{Verbatim}

\noindent Egyszer\H{u} import\'alas eset\'en (vagyis az {\tt import A} utas\'{\i}t\'as 
kiad\'asa ut\'an) a Python \'ertelmez\H{o} az abban l\'etrehozott elemeket k\"ul\"on 
t\'arter\"uletre rakja be, ezt szakkifejez\'essel {\sl n\'evt\'ernek} \index{n\'evt\'er} 
nevezz\"uk. A modul saj\'at n\'evtere teljesen elv\'alik a progam (szkript) saj\'at 
n\'evter\'et\H{o}l. Ez nagyon leegyszer\H{u}s\'{\i}tve k\'et dolgot jelent. Az egyik, hogy 
a fenti p\'eld\'at alapul v\'eve a f\H{o}programban is bevezethet\"unk egy {\tt text} 
v\'altoz\'ot, javaslatunk szerint csak indokolt esetben, mert nehezen felder\'{\i}thet\H{o} 
hib\'ak oka lehet.)

A m\'asik k\"ovetkezm\'enye, hogy a moduk egyes elemeire a n\'evt\'er sg\'{\i}ts\'eg\'evel 
tudunk hivatkozni, vagyis p\'eld\'aul:

\begin{Verbatim}[fontsize=\small]
print(A.text)"

i = A.foo(....)
\end{Verbatim}

\noindent Az ilyen neveket {\sl min\H{o}s\'{\i}tett neveknek} nevezz\"uk szakkifejez\'esel 
\index{min\H{o}s\'{\i}tett n\'ev}, hiszen ezzel tudjuk egy\'ertelm\H{u}en megmondani, mire 
gondolunk. P\'eld\'aul egy\'altal\'aban nem okoz gondot, ha k\'et k\"ol\"onb\"oz\H{o} modulban 
ugyanolyan nev\H{u} v\'altoz\'o van, mert ebben az esetben egyszer\H{u}en a {\tt A.text} \'es 
{\tt B.text} hivatkoz\'assal, a n\'evterek haszn\'alat\'avol pontosan a megfelel\H{o}t tudjuk 
el\'erni. (Programoz\'oi szemmel ez ugyanakkor teljesen rossz megk\"ozel\'{\i}t\'es a m\'asik 
f\'ele import\'alas miatt.)

A Python azonban lehet\H{o}v\'e teszi az import\'al\'as egy kiss\'e bonyolultabb, ugyanakkor a 
programoz\'o sz\'am\'ara nagyobb ellen\H{o}rz\'est ad\'o m\'odszer\'enek alkalmaz\'as\'at is.

\begin{Verbatim}[fontsize=\small]

from A import text

print(text)

\end{Verbatim}

\noindent A p\'elda alapj\'an egyb\H{o}l l\'atszik a k\'et legfontosabb k\"ul\"onbs\'eg. 
Az egyik, hogy az {\tt A} modulb\'ol csak egyetlen elem, a v\'altoz\'o lett import\'alva. 
A m\'asik, hogy ilyenkor nem haszn\'aljuk a min\H{o}s\'{\i}tett nevet, mintha a modul 
v\'altoz\'oja k\"ozvetlen\"ul a f\H{o}program n\'evter\'eben lett volna l\'etrehozva. Ez 
term\'eszetesen igaz lenne a f\"uggv\'enyekre \'es az oszt\'alyokra is.

Van ennek a k\'enyelmes el\'er\'esnek egy kellemetlen k\"ovetkezm\'enye is. Mivel (l\'atsz\'olag) 
nem j\"on l\'ere a moduloknak saj\'at n\'evtere, ez\'ert az azonos nev\H{u} elemek ugyan nem 
fognak \"ossze\"utk\"ozni egym\'assal de a programjaink olvashat\'os\'agat nagym\'ertekben rontj\'a 
az ilyen megold\'as. Esetlegesen az \'{\i}gy el\H{o}\'all\'o hib\'akat szint\'en nagyon neh\'ez 
elkapni \'es jav\'{\i}tani.

TODO teszt

-- Minta

Bonzolultabb programok, pontosabba szkriptek eset\'en nem mindegy, hogy a megold\'as sor\'an az 
azonos nev\H{u} modulok k\"oz\"ul melyik v\'alatoztot haszn\'aljuk. Ez f\H{o}leg akkor okozhat 
nem kis probl\'em\'at, ha nem egyed\"ul programozunk, ahogyan azt a f\"uggel\'ekben is olvasatjuk. 
Ez\'ert fontos lenne nyomon k\"ovetni azt, hogz a modulb\'ol milyen verzi\'ot haszn\'alunk, ehhez 
az al\'abbi \"otlet alkalmazhatjuk:

\begin{Verbatim}[fontsize=\small]
__version__ = "0.1.0" 
\end{Verbatim}

\subsection{Csak import\'aland\'o modulok}

A legalapvet\H{o}bb dolgokat a kor\'abbiak alapj\'an elsaj\'at\'{\i}tottuk, de nem 
\'art azokat ism\'etelve rendszerezni, illetve kiss\'e b\H{o}v\'{\i}teni. Ha egy 
modul kiz\'ar\'olag nem k\"ozvetlen\"ul v\'egrehajthat\'o utas\'{\i}t\'asokat tartalmaz, 
akkor azokat csak import\'al\'as ut\'an haszn\'alhatjuk.

Itt meg kell \'allnunk egy pillanatra, mert r\'eszletesebb \'ertelem\'esre szorul a "nem 
k\"ozvetlen\"ul v\'egrehajthat\'o utas\'{\i}t\'as" megfogalmaz\'as. A f\"uggv\'enyeket \'es 
oszt\'alyokat a Python t\'enylegesen l\'erehozza a mem\'ori\'aban, teh\'at azokat valamilyen 
szinten m\'egis v\'egrehajtja. Ugyanakkor l\'enyegesen m\'ashogy, mint ahogyan mondjuk egy 
\'ert\'ekad\'ast, vagy p\'eld\'aul egy {\tt print} f\"uggv\'enyh\'{\i}v\'ast. Ezek ugyanis 
haszn\'alj\'ak, esetleg meg is v\'altoztat\'ak az addig l\'etrehozott v\'altoz\'ok pilllanatnyi 
\'ert\'ek\'et, m\'{\i}g a f\"uggv\'enyek \'es oszt\'elyok k\'odj\'anak \'ertelmez\'ese nem. 
Ez\'ert a sz\'ohaszn\'alat tulajdonk\'eppen jogos \'es helyes. A programoz\'oi gyakorlatban 
mindezek ellen\'ere szoktak ilyenkor is haszn\'alni \'ert\'ekd\'asokat is, p\'eld\'aul:

\begin{Verbatim}[fontsize=\small]
# Modul Grafika.py

modul  = "Grafika"
verzio = "1.0.2.1"
datum  = "2020.07.12"	 

...

\end{Verbatim}

\noindent A fenti sorok egyszer\H{u}ek \'es j\'ol \'erthet\H{o}ek, m\'egis felmer\"ulhet az 
olvas\'okban, hogy mire is j\'o ez az eg\'esz. A v\'alasz el\'egg\'e k\'ezenfekv\H{o}, adott 
esetben seg\'{\i}t a modult (m\'eg import\'al\'as ut\'an is) azonos\'{\i}tani, mondjuk \'eppen 
nyomk\"ovet\'es c\'elj\'ab\'ol, amelyet a \ref{teszt}. fejezetben \'{\i}runk le r\'eszletesebben. 
Hasznos lehet m\'eg ezen k\'{\i}v\"ul dokument\'al\'asi c\'elb\'ol is, ak\'ar ig\'enybe vessz\"uk, 
ak\'at nem akarjuk ig\'enybe venni a \ref{dokument} fejezetben le\'{\i}rt Pythonos megold\'ast.

A r\'esz v\'eg\'en lenne m\'eg egy \'altal\'anosabb megjegyz\'es\"unk. Azt m\'ar minden olvas\'onk 
tudja a kor\'abbiak alapj\'an, hogy modulok alkalmaz\'as\'aval nem csak a programjainkat tudjuk 
olvashat\'obb\'a tenni, hanem egy-egy k\'odot m\'as szkriptekhez is fel tudunk haszn\'alni. Ez 
alapj\'an felmer\"ulhet a k\"ovetke\H{o} k\'erde\'es: mi va, ha a modulok importja egym\'asba 
   van \'agyazva. Legyen k\'et modulunk, {\tt A} \'es {\tt B}, \'es {\tt B} modul haszn\'alja a 
m\'asikat. Ha egy szkript import\'alja {\tt B}-t:

\begin{Verbatim}[fontsize=\small]
import B
\end{Verbatim}

\noindent akkor l\'etrej\"on a B n\'evt\'er a programban, a szok\'asos m\'odon min\H{o}s\'{\i}tett 
n\'evvel el\'erhet\H{o}, de A modul elemei nem \'erhet\H{o}k el k\"ozvetlen\"ul. Ugyanakkor a 
min\H{o}s\'{\i}tett n\'evvel meg tudunk h\'{\i}vni mondjuk egy A-beli f\"uggv\'enyt:

\begin{Verbatim}[fontsize=\small]
adat = B.A.fv(x)
\end{Verbatim}

\noindent Term\'eszetesen elk\'epzelhet\H{o}, hogy sz\"uks\'eg\"unk lenne az A modulra is, ilyenkor 
k\"ul\"on is beemelhetj\"uk.

\begin{Verbatim}[fontsize=\small]
import B
import A
\end{Verbatim}

\noindent Nem meglep\H{o}, ilyenkor mind a k\'et n\'evt\'er l\'etrej\"on, \'es valamilyen m\'odon 
B tartalmazza a teljes A modult.

%% Az import nem tranzitiv teljesen

TODO Teszt

\subsection{Import\'alhat\'o \'es v\'egrehajthat\'o modulok}

A programoz\'oi gyakorlat sor\'an nagyon sokszor kell olyan modult \'{\i}rnunk, amelyet ak\'ar 
mag\'aban is lehet ind\'{\i}tani, p\'eld\'ul tesztel\'esi c\'elb\'ol. Valamilyen m\'odon a 
Python \'ertelmez\H{o}nek tudnia kell, hogy a modult \'eppen import\'aljuk, vagy v\'egrehajtjuk. 
Erre a bels\H{o} \_\_name\_\_ v\'altoz\'o szolg\'al. Alkalmaz\'as\'ar\'ol l\'assunk egy 
egyszer\H{u} p\'eld\'at:

\begin{Verbatim}[fontsize=\small]
if __name__ == "__main__":
   # program
\end{Verbatim}

Abban az esetben, ha a modult k\"ozvetlen\"ul ind\'{\i}tjuk el, akkor a \_\_name\_\_ felveszi  
a "\_\_main\_\_" \'ert\'eket. Ilyen m\'odon az el\'agaz\'as \'atadja a vez\'erl\'est a modulban 
l\'ev\H{o} f\H{o}programnak. 

\section{A modulok \'es programok tesztel\'ese}
\label{teszt}

Az els\H{o} f\"uzetben m\'ar emlegett\"uk a k\'odok pr\'ob\'algat\'as\'at, hogy meg 
tudjuk vizsg\'alni egyszer\H{u}bb esetekben a m\H{u}k\"od\'es\"uket. Ez m\'eg nem 
jelent {\sl tesztel\'est}, ugyanis az minden esetben m\'odszeres elj\'ar\'as. Az 
alapvet\H{o}en azt jelenti, hogy a k\'od \'es a feladat ismeret\'eben el\H{o}zetesen 
\'atgondoljuk, hogy milyen esetekben fogjuk ellen\H{o}rizni az eredm\'enyeket. Nagyobb 
\'es bonyolultabb programok eset\'en nem csak a programot kell alaposan megtervezni, hanem 
mag\'at a tesztel\'est is. \index{teszt} Ennek ugyanis nem csak a feladata, hogy igazoljuk 
a k\'od helyes m\H{u}k\"od\'es\'et, hanem az esetleges hib\'ak ok\'at \'es felt\'etelezhet\H{o} 
hely\'et a k\'odban is seg\'{\i}tsen megtal\'alni.

Gondoljuk, hogy mindenki sz\'am\'ara vil\'agos, ehhez nagyon sok id\H{o} kell, n\'eha t\"obb 
is, mint a k\'od t\'enyleges meg\'{\i}r\'as\'ahoz. Nem v\'eletlen teh\'at az az elv\'ar\'as, hogy 
lehet\H{o}leg minn\'el nagyobb r\'esz\'et lehessen automatiz\'alni, \'{\i}gy azt r\'ab\'{\i}zhatjuk 
a sz\'am\'at\'og\'epre. Nem csup\'an id\H{o}t lehet ezzel megtakar\'{\i}tani, hanem alaposabb 
ellen\H{o}rz\'esnek is vethetj\"uk al\'a a programot. Ebben a fejezetben ezzel fogunk foglalkozni.

Sok-sok \'eves programoz\'oi tapasztalat alapj\'an tudjuk, hogy b\'armennyire is alaposan van egy 
szkript tesztelve, a sikeres teszt m\'eg nem jelenti azt, hogy nem maradtak benne hib\'ak. Csak 
arr'\'ol van sz\'o, hogy a fejleszt\'es sor\'an \'atgondolt esetekben a program helyesen m\H{u}k\"odik, 
\'es az \"onmag\'aban sem lebecs\"ulend\H{o} eredm\'eny. Senkinek a kedv\'et nem akarjuk elvenni, de 
a hib\'ak az al\'abiak lehetnek:

\begin{itemize}
   \item{a feladat nem lett eg\'eszen pontosan megfogalmazva}
   \item{a tesztel\'eskor kimaradtak ellen\H{o}rz\'esek}
   \item{a fejleszt\'eskor \'es tesztel\'eskor haszn\'alt k\"ornyezet elt\'er a futtat\'askor 
      haszn\'alt \'eles k\"ornyezett\H{o}l}
   \item{a programot nem teljesen arra haszn\'alj\'ak (haszn\'aljuk), amit a feladat tartalmaz.}
\end{itemize}

\noindent L\'atszik, hogy a programoz\'oi munka nem \'er v\'eget akkor, amikor a tesztel\'essel 
is k\'eszen vagyunk, \'es mindent hib\'atlannak tal\'alunk. Mint ahogyan a "puding pr\'ob\'aja az 
ev\'es", a haszn'\'alat sor\'an alaposabb tapasztalatot tudunk szerezni. Neh\'any h\'et, vagy h\'onap 
alatt nagy es\'ellyel j\"onnek el\H{o} az addig elb\'uj\'o probl\'em\'ak. Ilyenkor hozz\'a kell a 
k\'odhoz ny\'ulnunk, de mindig b\H{o}v\'{\i}teni kell a tesztesetek sz\'am\'at is.

\subsection{Alapoz\'as}

Bevezet\'esk\'ent sz\"uks\'eg van arra, hogy n\'eh\'any fogalmat a t\'enyleges 
tanulnival\'o el\H{o}tt le\'{\i}rjunk, ugyanis \'{\i}gy azt jobban fogjuk tudni 
meg\'erteni. Szint\'ugy fontosnak tartjuk, hogy az alapvet\H{o} elveket \'es 
m\'odszereket bemutassuk, lehet\H{o}v\'e t\'eve olvas\'oinknak az adott esetben 
a leghat\'ekonyabb megold\'asok kiv\'alaszt\'as\'at.

\subsubsection{Az alapfogalmak}

\noindent{\sl Teszt eset}: egy adott program-elem ellen\H{o}rz\'ese egy adott bemenet eset\'en. 
Fontos, hogy \"onmag\'aban is v\'egrehajthat\'o legyen. \index{teszt eset}

{\it Megjegyz\'es}: a tesztel\'esi folyamat legkisebb r\'esze.

\noindent{\sl Elv\'art eredm\'eny}: az adott teszt esetben az elfogadhat\'o esem\'eny, vagy kimeneti 
adat. Ha nem ez a tesztel\'es eredm\'enye, akkor az eset sikertelen. \index{elv\'art eredm\'eny}

\noindent{\sl Lefedetts\'eg}: annak m\'ert\'eke, hogy a program-elemek / k\'od / r\'eszfeladatok 
milyen r\'esze van teszttel ellen\H{o}rizve.

{\it Megjegyz\'es}: teljes lefedetts\'eg \'altal\'aban nem lehets\'eges, de t\"orekedni arra 
sz\"uks\'eges.

\subsubsection{Az alapelvek}

B\'armilyen m\'odon is tesztel\"unk az al\'abbi alapelveket mindenk\'eppen \'erdemes betartani:

\noindent{\sl Az egy\'ertelm\H{u}s\'eg elve}:

Alapvet\H{o}en fontos, hogy a teszt eset kimenetet / eredm\'enye egy\'ertelm\H{u}en mutassa, 
hogy sikeres volt-e, vagy sikertelen. 

{\it Megjegyz\'es}: ez akkor is \'erv\'enyes, ha nem automatikusan tesztel\"unk.

\noindent{\sl A megism\'etelhet\H{o}s\'eg elve}:

Eszerint az elv szerint ugyananaz a program-elem ism\'etelt teszt eset\'en ugyan\'ugy kell, 
hogy viselkedjen, ha id\H{o}kozben az a k\'od nem v\'altozott meg.

{\it Megjegyz\'es}: ez az elv is minden esetben \'erv\'enyes.

\noindent{\sl A teszt esetek f\"uggetlens\'eg\'enek elve}:

Egy program-elem tesztel\'ese sor\'an az esetek legyenek egym\'ast\'ol f\"uggetlenek, eredm\'eny\"uk 
ne f\"uggj\"on att\'ol, hogy milyen sorrendben hajtjuk \H{o}ket v\'egre. 

{\it Megjegyz\'es}: ez lehet\H{o}v\'e teszi, hogy esetleges hiba eset\'en annak ok\'at pontosabban 
meg lehessen \'allap\'{\i}tani. Indokoltan az\'ert ett\H{o}l el lehet tekinetni, b\'ar a hiba 
behat\'arol\'as\'at megnehez\'{\i}ti.

\noindent{\sl A biztons\'agos alkot\'oelemek elve}:

Ha a tesztelend\H{o} program-elem (legyen A) h\'{\i}v egy m\'asikat (B), akkor a B elemet kell el\H{o}bb 
tesztelni. 

{\it Magyar\'azat}: A tesztel\'esnek az is c\'elja, ahogyan m\'ar meg\'{\i}rtuk, hogy a hib\'ak hely\'et 
\'es ok\'at egyszer\H{u}bben megtal\'aljuk. Mivel a B alkot\'oelemet m\'ar kor\'abban letesztelj\"uk 
(\'es az esetleges hib\'akat kijav\'{\i}tjuk), ez\'ert az biztosan j\'ol m\H{u}k\"odik. Ez\'ert, ha 
hib\'at tal\'alunk, az csak A-ban lehet.

{\it Megjegyz\'es}: Nagyon gyakorlott tesztel\H{o}k ilyenkor a B elemet le szokt\'ak a tesztek idej\'ere 
cser\'elni mindig j\'ol "m\H{u}k\"od\H{o}" k\'odr\'eszletre.

\subsubsection{Az alapvet\H{o} m\'odszerek}

A tesztel\'es sor\'an a teszt eseteket a feladat (r\'eszfeladat) \'es a programk\'od alapj\'an kell 
meghat\'arozni. Ezt a kett\H{o}t nem egyforma ar\'anyban kell figyelembe venni, ilyen m\'odon alapvet\H{o}en 
k\'etf\'ele m\'odszert tudunk meghat\'arozni. Ezek a m\'odszerek att\'ol teljesen f\"uggetlenek, hogy a 
tesztel\'est k\'ezzel, vagy esetleg a sz\'am\'{\i}t\'og\'epre b\'{\i}zva automatiz\'altan csin\'aljuk.

\noindent{\sl A feh\'er doboz tesztel\'es}:
\index{feh\'er doboz teszt}

Minden esetet a tesztelend\H{o} k\'od alapos ismerete alapj\'an kell kital\'alni \'es meg\'{\i}rni. 
Biztos\'{\i}tani kell, hogy (lehet\H{o}leg) minden el\'agaz\'as minden \'ag\'ara r\'afusson legal\'abb 
egyszer a vez\'erl\'es. Minden ciklus eset\'eben legyen letesztelve mind a ciklusba val\'o bel\'ep\'es, 
mind pedig a ciklusb\'ol val\'o kil\'esp\'es. Ellen\H{o}rizni kell minden kiv\'etel eset\'et, amelyet 
az adott program-elemben helyileg lekezel a k\'od.

\noindent{\sl A fekete doboz tesztel\'es}:
\index{fekete dobboz teszt}

Kiindul\'asi pont alapvet\H{o}en a feladat, illetve r\'eszfeladat. Enn\'el a m\'odszern\'el nem kell 
ismerni a k\'odot. Az egyes teszt esetekn\'el a bemeneti adatokat a le\'{\i}r\'esnak megfelel\H{o}en 
kell \"ossze\'all\'{\i}tani. Sz\'amszer\H{u} adatok eset\'en mindenk\'eppen fel kell haszn\'alni a 
legkisebb \'es legnagyobb lehets\'eges \'ert\'ekeket. Figyelni kell arra, hogy a lehet\H{o} "legfurcs\'abb" 
bemeneti adatokat is haszn\'aljuk.

A k\'et bemutatott m\'odszer term\'eszetesen ide\'alis, a gyakorlatban sokszor hasznosabb a kett\H{o}t 
valamif\'ele m\'odon keverni. Ez azt jelenti, hogy alapvet\H{o}en az k\'od elv\'art viselked\'es\'et 
igyeksz\"unk ellen\H{o}rizni, de a megval\'os\'{\i}t\'o programk\'od ismeret\'eben. Iyenkor nem is 
szokott szempont lenni, hogy minden el\'agaz\'ast letesztelj\"unk, de a bonyultabb r\'eszekre 
mindenk\'eppen r\'a kell ker\"lnie a vez\'erl\'esnek. Sokszor a k\'odot \'es a teszteket nem ugyanaz a 
programoz\'o \'{\i}rja, ilyenkor j\'o, ha \H{o}k hat\'ekonyan \"ossze tudnak dolgozni.

\noindent{\sl Teszt vez\'erelt fejleszt\'es}:

L\'enyeg\'eben a feketedoboz m\'odszerhez hasonl\'{\i} egyetlen fontos k\"ul\"onbs\'eggel. El\H{o}sz\"or 
a fejleszt\'es sor\'an a teszteket kell a feladat le\'{\i}r\'asa alapj\'an meg\'{\i}rni, majd a k\'od 
elk\'sz\'{\i}t\'ese sor\'an arra figyelni, hogy az esetek egyre nagyobb sz\'amban fussanak hib\'atlanul. 
Ha net\'an a haszn\'alat sor\'an m\'egis tal\'alunk probl\'em\'at, akkor azt el\H{o}sz\"or a tesztekkel 
kell ism\'et el\H{o}\'all\'{\i}tani. \'Igy garant\'alhat\'o a pontos jav\'{\i}t\'as.

\subsection{A {\tt unittest} keretrendszer}
\index{unittest}

A Python programoz\'asi nyelv mostani nagy n\'epszer\H{u}s\'eg\'enek egyik oka \'eppen a be\'ep\'{\i}tett 
{\tt unittest} keretrendszer, amely nagym\'ert\'ekben megk\"onny\'{\i}ti az automatiz\'alt tesztek 
\'{\i}r\'as\'at. El\H{o}nye a f\"uggel\'ekben ismertetett {\tt doctest} modullal szemben az, hogy lehet\H{o}v\'e 
teszi egy-egy kisebb teszt ut\'olagos nagyobb egys\'egbe szervez\'es\'et. Az ellen\H{o}rz\H{o} utas\'{\i}t\'asok 
sz\'eles k\"or\'et bocs\'ajtja a programoz\'ok rendelkez\'es\'ere, ez\'ert bonyolultabb helyzetekben is j\'ol 
haszn\'alhat\'o. A r\'eszletesebb bemutat\'as \'es magyar\'azat el\H{o}tt l\'assunk egy egyszer\H{u} p\'eld\'at, 
m\'egpedig a kor\'abban m\'ar j\'ol ismert f\"uggv\'eny\"unkat:

\begin{Verbatim}[fontsize=\small]
# utest.py
import unittest

def paros(n):
   maradek = n % 2

   if maradek == 0:
      ki = 'I'
   else:
      ki = 'N'

   return(ki)

class TestFuggveny(unittest.TestCase):

    def testParos(self):
        res = paros(4)
        self.assertEqual(res, 'I', 'Nem paros')

    def testParatlan(self):
        res = paros(9)
        self.assertEqual(res, 'N', 'Nem paros')

if __name__ == "__main__":
   unittest.main()

\end{Verbatim}

\noindent A fenti k\'od el\'egg\'e egyszer\H{u} ahhoz, hogy a legfontosabb elemeket megismerj\"uk. De 
el\H{o}sz\"or n\'ezz\"uk meg, hogy mi t\"ort\'enik a fut\'askor:

\begin{Verbatim}[fontsize=\small]
python utest.py
..
----------------------------------------------------------------
Ran 2 tests in 0.000s

OK
\end{Verbatim}

\noindent A v\'egrahajt\'as sor\'an a keretrendszer automatikusan lefuttatja az eseteket, \'es mind a 
kett\H{o} sikeres volt. Kicsit t\"obbet tudunk meg, ha most is bekapcsoljuk "b\H{o}besz\'ed\H{u} 
\H{u}zemm\'odot":

\begin{Verbatim}[fontsize=\small]
python utest.py -v
testParatlan (__main__.TestFuggveny) ... ok
testParos (__main__.TestFuggveny) ... ok

----------------------------------------------------------------
Ran 2 tests in 0.000s

OK
\end{Verbatim}

\noindent A fut\'as sor\'an most pontosan nyomon tudjuk k\"ovetni, hogy milyen sorrendben 
hajtja v\'egre az eseteket. Tanuls\'agos lesz most a sz\'amunkra, hogy az egyik teszt esetet 
csak pr\'eba kedv\'e\'ert elrontjuk (a t\"obbi r\'eszlet teljesen megegyezik az el\H{o}z\H{o} 
p\'eld\'aval): 

\begin{Verbatim}[fontsize=\small]
class TestFuggveny(unittest.TestCase):

    def testParos(self):
        res = paros(4)
        self.assertEqual(res, 'I', 'Nem paros')

    def testParatlan(self):
        res = paros(9)
        self.assertEqual(res, 'N', 'Nem paros')
\end{Verbatim}

\noindent A szkriptet mind sim\'an, mind pedig sz\'osz\'aty\'ar m\'odban lefuttatjuk:

\begin{Verbatim}[fontsize=\small]

python utest_a.py
F.
================================================================
FAIL: testParatlan (__main__.TestFuggveny)
----------------------------------------------------------------
Traceback (most recent call last):
  File "utest_a.py", line 26, in testParatlan
    self.assertEqual(res, 'T', 'Nem paros')
AssertionError: 'N' != 'T'
- N
+ T
 : Nem paros

----------------------------------------------------------------
Ran 2 tests in 0.010s

FAILED (failures=1)

python utest_a.py -v
testParatlan (__main__.TestFuggveny) ... FAIL
testParos (__main__.TestFuggveny) ... ok

================================================================
FAIL: testParatlan (__main__.TestFuggveny)
----------------------------------------------------------------
Traceback (most recent call last):
  File "utest_a.py", line 26, in testParatlan
    self.assertEqual(res, 'T', 'Nem paros')
AssertionError: 'N' != 'T'
- N
+ T
 : Nem paros

----------------------------------------------------------------
Ran 2 tests in 0.010s

FAILED (failures=1)

\end{Verbatim}

\noindent Most n\'ezz\"uk meg egy kicsit r\'eszletesebben mag\'at a k\'odot. Ahhoz, hogy a keretrendszert az 
import\'al\'as ut\'an haszn\'alni tudjuk, az abban defini\'alt TestCase oszt\'alyb\'ol \"or\"okl\'essel 
l\'etre kell hoznunk egy saj\'at oszt\'alyt, \'es abban meg kell \'{\i}rni az ellen\H{o}rzend\H{o} 
eseteket met\'odusk\'ent, amelyeknek neve {\tt test}-el kell kezd\H{o}dj\"on. A keretrendszer ezeket 
fogja teszt eseteknek venni, \'es egym\'as ut\'an v\'egrehajtani. Az eredm\'eny ellen\H{o}rz\'es\'ere 
meg\"or\"ok\"olt\"unk n\'eh\'any met\'odust, most csak az {\tt assertEqual} nev\H{u}t haszn\'altuk. 
R\'oluk m\'eszletesebben is fogunk besz\'elni. A sikeres \'es sikertelen esetek k\"oz\"oti k\"ul\"onbs\'eg, 
hogy az ut\'obiakn\'al az ellen\H{o}rz\'es kiv\'etelt dob. Ezt nagyon \'erdemes megjegyezn\"unk, hiszen 
a teszt met\'odusok futtat\'asa sor\'an {\bf b\'armilyen} kiv\'etel hib\'anak sz\'am\'{\i}t. Az {\tt assertEqual}
met\'odus harmadik param\'etere t\'aj\'ekoztat\'o megjegyz\'es a tesztel\'esi hib\'ar\'ol, ak\'ar el is agyhat\'o.

Most az el\H{o}z\H{o} fejezethez hasonl\'oan n\'e\"zzuk meg, ha a tesztelend\H{o} modul tartalmaz egy egyszer\H{u} 
objektumot:

\begin{Verbatim}[fontsize=\small]
# utest1.py
import unittest

class Pont:
    def __init__(self, pX=0.0, pY = 0.0):
        self.x = pX
        self.y = pY

    def getX(self):
        return(self.x)

    def getY(self):
        return(self.y)

    def Mozgat(self, dX, dY):
        self.x += dX
        self.y += dY

class TestPontClass(unittest.TestCase):

    def setUp(self):
        self.egyik = Pont()
        self.masik = Pont(1,2.5)

    def tearDown(self):
        del self.egyik 
        del self.masik 

    def testAlap(self):
        adat = self.egyik.getX()
        self.assertEqual(adat,0.0)
        adat = self.egyik.getY()
        self.assertEqual(adat,0.0)

    def testMozgas(self):
        self.egyik.Mozgat(1.2,2.3)
        adat = self.egyik.getX()
        self.assertEqual(adat,1.2)
        adat = self.egyik.getY()
        self.assertEqual(adat,2.3)


    def testVegyes(self):
        adat = self.masik.getX()
        self.assertEqual(adat,1)
        adat = self.masik.getY()
        self.assertEqual(adat,2.5)


if __name__ == "__main__":
   unittest.main()

\end{Verbatim}

\noindent A futtat\'as sor\'an meggy\H{o}z\H{o}dhet\"unk arr\'ol, hogy a tesztek sikeresek. Alaposabban 
megn\'ezve ezt a k\'odot, azonnal tal\'alunk egy \'ujdons\'agot, a {\tt setUp} \'es a {\tt tearDown} 
met\'odusokat. Mind a kett\H{o} az \H{o}s {\tt unittest.TestCase} oszt\'alyban van eredetileg 
l\'etrehozva, de sz\"uks\'eg eset\'en a lesz\'armazott oszt\'alyunkban fel\"ul kell defini\'alni. Az 
els�H{o} \"ossze\'all\'{\i}tja, a m\'asodik pedig lebontja a tesztel\'eshez haszn\'alt k\"ornyezetet.
A {\tt setUp} lefut minden egyes teszt eset el\H{o}tt lefut, \'es a {\tt tearDown} minden egyes teszt 
eset ut\'an, m\'eg abban az esetben is, ha a teszt eset sor\'an valami kiv\'etel keletkezett. Ez azt 
is jelenti, hogy egy {\tt TestCase} oszt\'alyba azokat az eseteket kell egym\'assal  \"osszerakni, 
amelyek pontosan ugyanazt az el\H{o}k\'esz\'{\i}t\'est ig\'enylik.

\begin{figure}[h]
\setlength{\unitlength}{0.12in} % selecting unit length
\centering % used for centering Figure
\begin{picture}(32,24) % picture environment with the size (dimensions)
% 32 length units wide, and 15 units high.
\put(3,4){\framebox(6,3){tearDown}}
\put(4,7){\framebox(8,3){testMet\'odus2}}
\put(3,10){\framebox(6,3){setUp}}
\put(3,13){\framebox(6,3){tearDown}}
\put(4,16){\framebox(8,3){testMet\'odus1}}
\put(3,19){\framebox(6,3){setUp}}
\end{picture}
\caption{An LNL Block Oriented Model Structure} % title of the Figure
\label{fig:lnlblock} % label to refer figure in text
\end{figure}

A keretrendszer j\'o n\'eh\'any ellen\H{o}rz\H{o} met\'odust bocs\'ajt a rendelke\'es"unkre, itt most 
csak a legfontosabbakat fogjuk ismertetni. Sz\"uks\'eg eset\' mi magunk is tudunk ilyet \'rni. L\'assuk 
a met\'odusokat:

\begin{center}
   \begin{tabular}{ll}
   {\bf Met\'odus} & {\bf Mit tesztel} \\
      {\tt assertEqual} & A k\'et elem egyenl\H{o}. \\
      {\tt assertNotEqual} & A k\'et elem egyenl\H{o}. \\
      {\tt assertTrue} & A logikai kifejez\'es igaz. \\
      {\tt assertFalse} & A logikai kifejez\'es hamis. \\
      {\tt assertIn} & Az elem benne van a list\'aban. \\      
   \end {tabular}
\end{center}

\noindent Term\'eszetes, a magyar\'az\'o sz\"oveg param\'etert a ahiba jelleg\'er\H{o}l b\'armelyik 
eset\'eben alkalmalmazhatjuk.

M\'eg a k\'es\H{o}bbiekben m\'elyebbre ia fogunk hatolni a {\tt unittest} keretrendszer haszn\'alat\'anak 
rejt\'elmeibe, most k\'et r\"ovid megjegyz\'est szeretn\'enk itt tenni. Az egyik, hogy szkriptben t\"obb 
{\tt TestCae} lesz\'armazott is lehet, mindegyik teszt eseteit le fogja a Python futtatni. A m\'asik, 
hogy sz\"ks\'eg eset\'en be is tudjuk "\'egetni" a szkriptbe (modulba), hogy minden futtat\'as eset\'en 
   r\'eszletes inform\'aci\'ot kapjunk, csak a f\H{o}program r\'eszt kell egy kicsit \'at\'{\i}rni:

\begin{Verbatim}[fontsize=\small]
if __name__ == "__main__":
   unittest.main(verbosity=2)
\end{Verbatim}

\noindent Ez pontosan olyan, mintha a futtat\'asn\'al haszn\'altuk volna a {-v} kapcsol\'ot.

Term\'eszetesen egy oszt\'aly komolyabb tesztel\'ese eset\'en k\"onnyen elk\'epzelhet\H{o}, hogy nem 
csak egyetlen {\tt TestCase} lesz\'armazott lehet egy szkriptbben, hanem t\"obb is. Mivel a {\tt setUp} 
\'es a {\tt tearDown} met\'odusok minden egyes teszt eset el\H{o}tt \'es ut\'an k\"ul\"o lefutnak, oda 
kell figyelni arra, hogy egy oszt\'alyba olyan teszt met\'odusokat rakjunk, amelyeket ugyan\'ugy kell 
el\H{o}k\'esz\'{\i}teni. Ezeket a met\'odusokat a keretrendszer minden alkalommal ugyanolyan sorrendben 
fogja lefuttatni. A k\'es\H{o}bbiekben be fogjuk mutatni, hogy mik\'ent is lehet ezt megv\'altoztatni. 

\subsection{Tesztel\'es magasfokon}

Az alapok megismer\'ese ut\'an itt az ideje, hogy kicsit bonyolultabb eseteket is megn\'ezz\"unk, 
egy-k\'et \'erdekesebb megold\'ast is megismerj\"unk. A hely hi\'any\'aban most sem fogunk minden 
r\'eszletre kit\'erni, n\'eh\'any j\'ol \'es rugalmasan alkalmazhat\'o \"otletet megmutatunk. 
Ha valaki enn\'el m\'elyebb ismeretkre v\'agyik, akkor a f\"uggel\'ekben tal\'unk utal\'ast 
olvasnival\'okkal kapcsolatban.

Az egyik feljebb le\'{\i}rt p\'eld\'an l\'atjuk, hogy a tesztelend\H{o} k\'od \'es az automatikus 
teszt esetek egyetlen f\'ajlban vannak. Bonyolultabb programok eset\'eben ez egyr\'eszt teljesen 
\'attekinthetetlenn\'e teszi az eg\'esz k\'odot, m\'asr\'eszt pedig nem is biztos, hpgy szerencs\'es 
az "\'eses" futtat\'as k\"ornyezet\'et a teszt esetekkel \"ossekusz\'alni. T\'etelezz\"uk fel, hogy 
teszteln\"unk kell a {\tt modulA.py} \'allom\'anyban tal\'alhat\'o f\"uggv\'enyeket, illetve 
obejktumokat. Ilyenkor szok\'asos megold\'as, hogy ehhez meg\'{\i}runk a {\tt testA.py} tesztel\H{o} 
szkriptet, amely legal\'abb az al\'abbiakat tartalmazza:

\begin{Verbatim}[fontsize=\small]
import unittest

import modulA

....
\end{Verbatim}

\noindent Ha sz\"uks\'eg van egy\'eb modulokra is, akkor term\'eszetesen azokat is bele kell \'{\i}rni. 
Ennek a megold\'asnak az az el\H{o}nye, hogy csin\'alhatunk egy k\"ul\"on k\"onyvt\'arat a fejleszt\'es 
\'es tesztel\'es c\'elj\'ara, egy m\'asikat pedig a haszn\'atnak. (Az import\'assal kapcsolatos b\H{o}vebb 
ismereteket egy kor\'abbi fejezetben tal\'aljuk.) 

A programoz\'oi gyakorlatban id\H{o}nk\'ent el\H{o}fordul az, hogy az automatiz\'alt tesztel\'es sor\'an 
egy-egy teszt esetet \'eppen ne hajtsunk v\'egre, de ugyanakkor nem lenne hasznos a k\'odb\'ol kivenni. 
Szerencs\'ere erre van egyszer\H{u} megold\'as, a {\tt unittest} keretrendszerben meg tudjuk jel\"olni az 
\'atmenetileg felesleges eseteket. L\'assuk a p\'eld\'at: \label{skiptest}

\begin{Verbatim}[fontsize=\small]
# skiptest.py
import unittest

def paros(n):
   maradek = n % 2

   if maradek == 0:
      ki = 'I'
   else:
      ki = 'N'

   return(ki)

class TestFuggveny(unittest.TestCase):

    def testParos(self):
        res = paros(4)
        self.assertEqual(res, 'I', 'Nem paros')

    def testParatlan(self):
        res = paros(9)
        self.assertEqual(res, 'N', 'Nem paros')

    @unittest.skip("Ugorjunk")
    def testTort(self):
        res = paros(4)
        self.assertEqual(res, 'I', 'Nem paros')

if __name__ == "__main__":
   unittest.main()
\end{Verbatim}

\noindent A fenti k\'od alapj\'an l\'atszik, elegend\H{o} volt nem sz\"uks\'eges {\tt test... } met\'odust 
a megel\H{o}z\H{o} {\sl dekor\'atorral} megjelel\"olni. Futtat\'as eset\'en:

\begin{Verbatim}[fontsize=\small]
python skiptest.py
..s
----------------------------------------------------------------------
Ran 3 tests in 0.000s

OK (skipped=1)
\end{Verbatim}
 
\noindent L\'atjuk, az \"osszefoglal\'oban csak annyi jelent meg, hogy az eset \'at lett ugorva. R\'eszletesebb 
kimenet a szok\'as szerint:

\begin{Verbatim}[fontsize=\small]
python skiptest.py -v
testParatlan (__main__.TestFuggveny) ... ok
testParos (__main__.TestFuggveny) ... ok
testTort (__main__.TestFuggveny) ... skipped 'Ugorjunk'

----------------------------------------------------------------------
Ran 3 tests in 0.000s

OK (skipped=1)
\end{Verbatim}

\noindent Ebben a list\'aban nem csup\'an azt vehetj\"uk \'eszre, hogy az esetet \'atugrottuk, hanem ennek az oka is 
j\'ol l\'atszik. Van lehet\H{o}s\'eg arra is, hogy az egyes eseteket felt\'etelesen is \'atugorhassuk, de ennek a 
le\'{\i}r\'asnak a keret\'eben ezzel tarjedelmi okokb\'ol ezzel nem foglalkozunk. 

Bonyolultabb, t\"obb mudult is haszn\'al\'o szkriptek eset\'eben el\H{o} szokott fordulni, hogy a teszt eseteket nem 
egy \'allom\'anyban vannak meg\'{\i}rva. Ilyenkor ezeket mindig k\'ezzel kell elind\'{\i}tani, de szerencs\'ere a 
Python nyelven lehets\'eges, hogy ezeket egybefoglaljuk. Erre szolg\'al a keretrendszer {\tt TestLoader} oszt\'aly. 
Ennek seg\'{\i}ts\'eg\'evel be tudjuk t\"olteni a k\"ul\"on meg\'{\i}rt forr\'asokb\'ol a teszteket.

TODO


-- MOCK ??

Kor\'abban m\'ar eml\'{\i}tett\"uk, hogy a tesztel\'es sor\'an el\H{o}sz\"or a legink\'abb be\'agyazott 
alkot\'oelemet kell ellen\H{o}rizni, hogy egy\'ertelm\H{u}bben lehesen az esetleges hib\'akat behat\'arolni. 


\section{Nyomk\"ovet\'es Python m\'odra}
\label{trace}

Az el\H{o}z\H{o} f\"uzetben m\'ar volt arr\'ol sz\'o, hogy id\H{o}nk\'ent sz\"uks\'eg lehet a programok (szkriptek) 
fut\'as\'anak k\"ovet\'es\'ere. Erre ott a {\tt print} utas\'{\i}t\'ast hasn\'altuk, ami a k\'erny\H{o}re \'{\i}rta 
ki az \"uzeneteket. Ez akkor elegend\H{o} is volt a sz\'amunkra, mert viszonylag egyszer\H{u} \'es el\'egg\'e r\"ovid 
programok eset\'eben alkalmaztuk. Hosszabb \'es bonyolultabb programokn\'al azonban nem megfelel\H{o}, mert a sok 
kimenet \'{\i}gy teljesen \'attekinthetetlen. A szok\'asos programoz\'oi \"otlet ilyenkor, hogy ezeket a sorokat 
egy \'allom\'anyba iratjuk ki, ezt {\sl napl\'o \'allom\'anynak}, vagy szakkifejez\'essel {\sl lognak} nevezz\"uk. 
\index{napl\'o} \index{log} 

Ebben a fejeztben t\"obb m\'odszert is be fogunk mutatni, hogyan lehet ezt a probl\'em\'at megoldani. Annak 
ellen\'ere, hogy a Python-nak van egy szabv\'anyos napl\'oz\'o modulja, fontosnak tartjuk, hogy az olvas\'oink 
t\'enylegesen kipr\'ob\'alj\'ak a t\"obbf\'ele \"otletet is. A be\'ep\'{\i}tett modul ugyanis nagyon rugalmasan 
alkalmazhat\'o, de bizonyos esetekben t\'uls\'agosan is k\"or\"ulm\'enyes lehet. Nem minden feladat eset\'en 
kell alkalmaznunk azt a sokoldal\'u modult, n\'eha elegend\H{o} ann\'al egyszer\H{u}bbb is. Nem mellesleg a t\"obbi 
m\'odszer meg\'{\i}r\'asa j\'o programoz\'asi gyakorlatot is jelenthet mindenkinek.

Miel\H{o}tt azonban ismertetn\'enk a lehets\'eges m\'odszereket, szeretn\'enk a nyom\-k\"o\-ve\-t\'es\-sel 
kapcsolatosan 
a legfontosabb szempontokat tiszt\'azni. Ahhoz, hogy ki tudjuk v\'alasztani az adott esetben alkalmazand\'o 
legjobb megold\'ast, vil\'agosan tudnunk kell, hogy a napl\'o \'allom\'anyt mire is akarjuk haszn\'alni, 
mi a {\bf c\'elja}. A programoz\'oi gyakorlatban alapvet\H{o}en k\'etf\'ele ok miatt kellheth a log. Ha a 
{\bf hib\'ak } megtal\'al\'as\'ara, az okai kider\'{\i}t\'es\'ere akarjuk haszn\'alni, azt els\H{o}sorban 
a {\bf fejleszt\'es}, illetve a {\bf tesztel\'es} alkalm\'aval vessz\"uk ig\'enybe. A fontos esem\'enyek 
(hib\'ak) t\'arol\'as\'ara viszont akkor van m\'ar sz\"uks\'eg, amikor a programot m\'ar {\bf "\'eles"} 
esetekben futtatjuk. A k\'etfele c\'el m\'as-m\'as k\"ovetelm\'enyeket jelent.\footnote{Ezek a szempontok 
m\'as programoz\'asi nyelv eset\''eben is ugyan\'ugy \'erv\'enyesek, \'erdemes teh\'at mejegyezni.}

\noindent{\sl Fejleszt\'esi / tesztel\'esi szakasz}:

A fejlesztes \'es \'es tesztel\'es sor\'an a legfontosabb, hogy a log alapj\'an r\'eszletesen is ellen\H{o}rizni 
tudjuk a fut\'as menet\'et. Ehhez az al\'abbiakra van sz\"uks\'eg, hogy a napl\'o:

\begin{itemize}
   \item{{\tt --} tartalmazzon b\H{o}s\'eges inform\'aci\'ot,}
   \item{{\tt --} az esetek t\"obbs\'eg\'eben elegend\H{o}, ha az utols\'o fut\'as adatait tartalmazza,} 
   \item{{\tt --} legyen nagy felbont\'as\'u, vagyis sokszor irassunk bele adatokat,} 
   \item{{\tt --} mutassa vil\'agosan a v\'egrehajt\'as menet\'et, vagyis minden el\'agaz\'as \'es ciklus eset\'en 
      a fontos v\'altoz\'ok \'ert\'eke ker\"ulj\"on bele.}
\end{itemize} 

\noindent A fentiek alapj\'an szerint\"unk minden olvas\'onk sz\'am\'ara vil\'agos, hogy ezekkel 
a felt\'etelekkel r\"ovid id\H{o} alatt is el\'egg\'e sok sor ker\"ulhet be a nyomk\"ovet\'esre 
haszn\'alter \'allom\'anyba. Ez term\'eszetes velej\'ar\'oja az ilyen logol\'asnak, hiszen a 
legfontosabb dolog, hogy a lehet\H{o} legpontosabban be lehessen hat\'arolni az esetleg hib\'ak 
hely\'et \'es ok\'at. Ugyanakkor \'eppen ez az, amit miatt az igazi, "\'eles", haszn\'alatban 
nem alkalmaz. Mivel ez ut\'obbi esetben az adatok t\'arol\'al\'asa hosszabb id\H{o}n kereszt\"ul 
a legfontosabb szempont. Term\'eszetesen felmerer\"ulhet, hogy a k\'esz programban a megfelel\H{o} 
"b\H{o}besz\'ed\H{o}" sorokat ki\'{\i}r\'o utas\'{\i}t\'asok kit\"or\"olhet\H{o}k, de sajnos ez 
el\'egg\'e macer\'as munka, \'es okozhat v\'aratlan hib\'akat. Vannak persze gyakorlatiasab megold\'asok 
is, de azok el\H{o}zetes tertve\'est, \'es programoz\'o munk\'at ig\'enyelnek. N\'eh\'any ilyen \"otletet 
be is fogunk mutatni.

\noindent{\sl Haszn\'alati szakasz}:

Ebben az \"uzemm\'odban sok\'aig, ak\'ar hetekig is kell gy\H{u}jteni az adatokat, ez\'ert:

\begin{itemize}
   \item{{\tt --} a napl\'o legyen t\"om\"or,}
   \item{{\tt --} a napl\'o minden esetben a kor\'abbi adatokhoz f\H{u}zz\"on hozz\'a}
   \item{{\tt --} a napl\'o csak a legfontosabb adatokat tartalmazza,}
   \item{{\tt --} a futtat\'as kimenete, illetve eredm\'enyess\'ege a napl\'o alapj\'an 
      egy\'ertel\-m\H{u}\-en \'ertekelhet\H{o} legyen, ak\'ar ut\'olag is}
   \item{{\tt --} a logb\'ol minden esetben der\"ulj\"on ki az esetlegesen el\H{o}fordul\'o hib\'ak 
      s\'ulyoss\'aga}
\end{itemize} 

\noindent Gondoljuk, hogy nagyj\'ab\'ol mindenki l\'atja, ez a lista szinte pontosan ellent\'ete az 
el\H{o}z\H{o}nek. Ugyanakkor b\'armikor lehet olyan hiba, amely miatt a program k\'odj\'ahoz hozz\'a 
kell k\'es\H{o}bb ny\'ulni. Ilyenkor viszont a log nem fog el\'eg seg\'{\i}ts\'eget ny\'ujtani a 
hiba hely\'enek \'es ok\'anak meghat\'aroz\'as\'ahoz. A kor\'abbi megjegyz\'es alapj\'an azonban egy 
kis tov\'abbi programoz\'oi munk\'aval b\'armikor ki- \'es bekapcsolhatjuk a r\'eszletesebb napl\'oz\'ast. 

\subsection{Napl\'o \'allom\'any \'{\i}r\'asa}

Most kezdj\"uk el a kor\'abban meghat\'arozott feladat, pontosabban feladatok megold\'as\'ahoz vezet\H{o} 
legegyszer\H{u}bb utat kitaposni, vagyis a futtat\'as sor\'an k\'esz\'{\i}ts\"unk egy napl\'o \'allom\'any.
Ehhez nagym\'ert\'ekben sz\"uks\'eg\"unk lesz a \ref{file}. fejezetben olvasottakra, hiszen egyenesen egy 
\'allom\'anyba fogjuk k\"uldeni az adatokat. Vagyis azt az \'allom\'anyt meg is kell nyitni. A kor\'abbi 
ismereteknek megfelel\H{o}en k\'etf\'ele m\'odon tudjuk ezt megtenni, \'{\i}r\'asra, vagy hozz\'af\H{u}z\'esre. 
A kett\H{o} k\"oz\"otti k\"ul\"onbs\'eg kicsinek t\H{u}nik, de val\'oj\'aban nem az. A napl\'oz\'as 
szempontj\'ab\'ol az elt\'er\'est az al\'abbi t\'abl\'azatban tudjuk \"osszefoglalni:

\begin{center}
   \begin{tabular}{cl}
   {\bf M\'od} & {\bf Jelent\'es} \\
   {\tt w} & A napl\'o \'allom\'any kor\'abbi tartalma t\"orl\H{o}dni fog. \\
   {\tt a} & A napl\'o kor\'abbi tartalma megmarad. \\
   \end {tabular}
\end{center}

Ahogyan az \'altal\'anos bevezet\H{o}ben eml\'{\i}tett\"uk, az els\H{o} esetet els\H{o}sorban a tesztel\'es, 
illetve fejleszt\'es sor\'an alkalmazhatjuk. A m\'asodik pedig seg\'{\i}t abban, hogy a napl\'o tartalm\'at 
ak\'ar hosszabb ideig is megtarthassuk. De ez csak a program elej\'en t\"ort\'en\H{o} megnyit\'as alkalm\'aval 
fontos, ugyanis kor\'abban a file-ok haszn\'alat\'aval foglalkoz\'o fejezetben bemutatottakkal \"osszhangban 
azt kell mondanunk, hogy nem biztons\'agos egy \'allom\'anyt \'{\i}r\'askor hosszabb ideig nyitva tartani. 
Teh\'at minden egyes sor ut\'an c\'elszer\H{u} a kimenetet minden alkalommal azonnal lez\'arni. L\'assunk erre 
egy egyszer\H{u} p\'eld\'at:

\begin{Verbatim}[fontsize=\small]
# elso alkaloma
naplo = open("futas.txt", "w")
naplo.write('Start\n')
naplo.close()

 ......

# sokadik alkaloma
naplo = open("futas.txt", "a")
naplo.write('Itt van.\n')
naplo.close()
\end{Verbatim}

\noindent A r\"ovid p\'elda egyszer\H{u}en magyar\'azza az \"otletet. Minden egyes sor ki\'{\i}r\'asa 
ut\'an az adatveszt\'es elker\"ul\'ese \'erdek\'eben lez\'arni a file-t. Logikus, hogy a m\'asodik 
sort\'ol kezdve mindig hozz\'af\H{u}z\'esre kell nyitni az \'allom\'anyt. 

Ezen a ponton sz\"uks\'eg van arra, hogy n\'eh\'any tov\'abbi megjegyz\'est tegy\"unk. Egy nagyobb programn\'al 
A fenti m\'odon sok felesleges k\'odsort kellene \'{\i}rni, c\'elszer\H{u}nek l\'atszik, hogy a nagyobb 
k\'enyelem kedv\'e\'ert egy k\"ul\"on f\"uggv\'enybe tegy\"uk a napl\'oz\'ast:

\begin{Verbatim}[fontsize=\small]
def Log(naploFile, text, kezdet = 0):
   if kezdet != 0:
      trc = open(naplo,"w")
   else:
      trc = open(naplo,"a")
   trc.write('%s\n'%text)
   trc.close()
\end{Verbatim}

\noindent Azt gondoljuk, hogy a fenti k\'odhoz sok magyar\'azatot nem kell f\H{u}zni, a meg\'ert\'es\'ehez 
sz\"uks\'eges ismeretek a kor\'abbi f\"uzet\"unkben benne vanna. K\'et dolgot azonban szertn\'enk megjegyezni. 
Az egyik, hogy a kezdeti kiirat\'ast, \'es a t\"obbit "begy\'urtuk" ebbe a f\"uggv\'enybe, erre szolg\'al 
a bevezettett {\tt kezdet} nev\H{u} patam\'eter. A m\'asik, hogy a napl\'o nev\'et nem "huzaloztuk" be a 
f\"uggv\'enybe, \'{\i}gy sokkal rugalmasabban haszn\'alhat\'o. Esetleg m\'eg k\"ul\"on modulba is tehetj\"uk, 
t\"obb programban is haszn\'alhatjuk. Ez egys\'egess\'e teszi a logokat, ez a programoz\'oi tapasztalatunk 
alapj\'an nagyon meg tudja k\"onny\'{\i}teni fut\'asok ellen\H{o}r\'es\'et. Haszn\'alata p\'eld\'aul:

\begin{Verbatim}[fontsize=\small]
# elso alkaloma
Log("futas.txt",'Start', kezdet=1)
 ......

# sokadik alkaloma

Log("futas.txt",'Itt van.')
\end{Verbatim}

\noindent A k\'enyelem kedv\'e\'ert m\'eg a k\"otelez\H{o} soremel\'est is beillesztett\"uk a f\"uggv\'enybe. 
Ahogyan a k\'et legut\'obbi k\'odot n\'ezz\"uk, vil\'agos a napl\'o \'allom\'any nevet minden esetben meg 
kell adnunk, de erre k\"onnyen felhaszn\'alhatunk egy v\'altoz\'ot:

\begin{Verbatim}[fontsize=\small]
logf = "futas.txt"
# elso alkaloma
Log(logf,'Start', kezdet=1)
 ......
\end{Verbatim}

\noindent Ez r\'eszben tov\'abb egyszer\H{u}s\'{\i}ti a helyzet\"unket, ugyanakkor azonban az \'allom\'any 
nev\'et a programban szerepl\H{o} \"osszes f\"uggv\'enynek \'es \"osszes oszt\'alynak \'at kell adni. A 
napl\'oz\'o f\"uggv\'eny\"unket ki lehetne b\H{o}v\'{\i}teni sok mindennel, az al\'abbi felsorol\'assal 
adunk n\'eh\'any \"otletet:

\begin{itemize}
   \item{{\tt -} d\'atumot \'es a pontos id\H{o}t,}
   \item{{\tt -} a Python forr\'as \'allom\'any nev\'et,}
   \item{{\tt -} a Python forr\'as sorsz\'am\'at.}      
\end{itemize} 

\noindent Mindegyikre val szabv\'anyos lehet\H{o}s\'eg a Python programoz\'asi nyelven. Arra is felh\'{\i}vjuk 
a figyelmet, hogy ugyan nagyon sokat tanulhatunk a Python programoz\'as titkaib\'ol, viszont tudunk enn\'el 
k\'enyelmesen haszn\'alhat\'o logol\'ast is.

\subsection{Napl\'oz\'o oszt\'aly \'{\i}r\'asa}

Folytatva az el\H{o} r\'esz utols\'o gondolat\'at, term\'eszetesen mer\"ul f\"ol az emberben, hogy a napl\'oz\'asra 
is haszn\'alhatunk objektumot, \'es nem is kell sokat dolgozni egy rugalmasan \'es k\'enyelmes haszn\'alhat\'o 
oszt\'aly\'ert. Az eg\'esz megold\'as nem lesz tulajdonk\'eppen m\'as, mint be\'agyazni a f\"uggv\'enyt egy 
oszt\'alyba, term\'eszetesen a megfelel\H{o} k\"or\'{\i}t\'essel. Az egy\'ertelm\H{u}, hogy a k\'odol\'as 
megked\'ese el\H{o}tt \'at kell gondolni, hogy mi mindent szeretn\'enk az oszt\'allyal elv\'egeztetni. N\'ah\'any 
szempont:

\begin{itemize}
   \item{{\tt -} a napl\'o \'allom\'any nev\'et a konstruktorban kell megadni,}
   \item{{\tt -} a pontos id\H{o}t mindenk\'eppen ki kell \'{\i}rni,}
   \item{{\tt -} a logol\'ast lehessen ki- \-es bekapcsolni k\'enyelmesen.}      
\end{itemize} 

\noindent Term\'eszetesen mer\"ulnek fel tov\'abbi \"otletek is, azok egy r\'esz\'et fokozatosan fogjuk 
be\'ep\'{\i}teni a k\'odba. Most l\'assunk egy egyszer\H{u}, lehet\H{o} legminim\'alisabb megold\'ast 
a probl\'em\'ara:

\begin{Verbatim}[fontsize=\small]
## 1. valtozat

class Naplo:
    def __init__(self, pNeve, pMeret, pMode = 'a'):
        self.logNev = pNeve
        self.meret = pMeret
        self.startMode = pMode
        self.mode = 'a'
        self.sor = 0
        self.puffer = []
        self.start = True

    def Lezar(self):
        if self.sor != 0:
            if self.start:
                trc = open(self.logNev,self.startMode)
            else:
                trc = open(self.logNev,self.mode)
            for idx in range(self.sor):
                trc.write(self.puffer[idx])
            trc.close()

    def Kiir(self, txt):
        msg = '%s\n'%txt
        self.puffer.append(msg)
        self.sor += 1
        if self.sor == self.meret:
            print(' .. logName = %s'%self.logNev)
            if self.start:
                trc = open(self.logNev,self.startMode)
            else:
                trc = open(self.logNev,self.mode)
            for idx in range(self.sor):
                trc.write(self.puffer[idx])
            trc.close()
            self.start = False
            self.puffer = []
            self.sor = 0
\end{Verbatim}

\noindent A fenti k\'od nem olyan bonyolult, de egy-k\'et megjegyz\'est az\'ert szeretn\'enk 
hozz\'af\H{u}zni. A legfontosabb k\'erd\'es, hogy az el\H{o}z\H{o} r\'eszben l\'atott  
f\"uggv\'ennyel ellent\'etben az itt bemutatott met\'odus mi\'ert nem \'{\i}rja ki azonnal 
a logba a sort. Erre a v\'alasz egy kicsit a futtat\'as r\'eszleteiben rejlik, hiszen a 
t\'enyleges \'allom\'any-m\H{u}veletek az esetek t\"obbs\'eg\'eben el\'egg\'e lass\'uak a 
t\"obbi m\H{u}velethez k\'epest. Ez\'ert id\H{o}ben hat\'ekonyabb lesz a k\'od, ha n\'eh\'any 
sort a t\'enyleges ki\'{\i}r\'as el\H{o}tt egy pufferben \"osszegy\H{u}jt\"unk, \'es egyszerre 
tessz\"uk ki az f\'ajlba. Ennek a bels\H{o} t\'arol\'onak a m\'eret\'et a konstruktorban 
\'all\'{\i}tjuk be, k\"ozvetlen\"ul a log \'allom\'any megad\'asa ut\'an. (Gyakorlatban nem 
c\'elszer\H{u} nagym\'eret\H{u} puffert v\'alasztani, 30-40 sor elegend\H{o} kell legyen.) A 
haszn\'alat sor\'an viszont k\"onnyen elk\'epzelhet\H{o}, hogy a program v\'egrhajt\'asa 
v\'eg\'en maradnak az objektumban ki nem \'{\i}rt adatok. Ezt a {\tt Lezar} met\'odus elint\'ezi, 
amit a szkript v\'ege el\H{o}tt meg kell h\'{\i}vni. (Felmer\"ulhet, hogy ezt a k\'odot az 
oszt\'aly destruktor\'aba tessz\"uk, de akkor nem egy\'ertelm\H{u} napl\'oz\'as lez\'ar\'asa. 
Emellett a destruktorral kapcsolatoan vannak technikai neh\'ezs\'egek, ezt k\'es\H{o}, a 
\ref{destruktor}. fejezetben r\'eszletezz\"uk.) A m\'ar kor\'abban megismert egyszer\H{u} esetben 
n\'ezz\"uk meg a haszn\'alat\'at:

\begin{Verbatim}[fontsize=\small]
trc = Naplo('futas.txt', 10)

# elso alkalom
trc.Kiir('Start')
 ......

# sokadik alkalom
trc.Kiir('Itt van.')
trc.Lezar()
\end{Verbatim}

\noindent L\'atszik a k\'odb\'ol, hogy a napl\'oz\'o objektum l\'etrehoz\'asakor a konstruktorban nem 
adtuk meg megnyit\'asi m\'odott, ez\'ert m\'ar a legels\H{o} alkalommal is hozz\'af\H{u}z\'esre nyitjuk 
meg az \'allom\'anyt. (Ahogy ez az oszt\'aly defin\'{\i}ci\'oja alapj\'an is l\'atszik, ez az alap\'ertelmezett 
m\H{u}k\"od\'es.) A ki\'{\i}r\'o met\'odus (\'es a destruktor is) nymon tudja k\"ovetni, hogy t\"ort\'ent-e 
t\'enyleges \'{\i}r\'as a f\'ajlba. A puffer m\'eret\'et {\tt 10} sorban hat\'aroztuk meg, ami egy biztons\'agos 
m\'eret, \'es nem ritk\'{\i}tja meg nagyon a sorok ment\'es\'et.

Az el\H{o} r\'eszben tal\'alunk egy list\'at, hogy a logba milyen adatokat szeretn\'enk m\'eg belevenni. Most meg 
fogjuk oldani, hogy a Python forr\'as modul neve is t\'arolva legyen a napl\'oban. Nem fogjuk a terljes k\'odot 
megadni, csak a v\'altoz\'asokat, a megfelel\H{o} szkript \"ossze\'all\'{\i}t\'asa az ovas\'ok dolga lesz.

\begin{Verbatim}[fontsize=\small]
## tesztelni
## 1. plussz

def Kiir(self, forras, txt):
   msg = '%s, %s\n'%(forras,txt)
   self.puffer.append(msg)
   self.sor += 1
   if self.sor == self.meret:
      if self.start:
         trc = open(self.logNev,self.startMode)
      else:
         trc = open(self.logNev,self.mode)
      for idx in xrange(self.meret):
         trc.write(self.puffer[idx])
         trc.close()
         self.start = False
         self.puffer = []
         self.sor = 0

\end{Verbatim}

\noindent Ennek haszn\'alata:

\begin{Verbatim}[fontsize=\small]
trc = Naplo('futas.txt', 10)

# elso alkalom
trc.Kiir(__file__,'Start')
 ......
 trc.Lezar()
\end{Verbatim}

\noindent Az egyetlen apr\'onak t\H{u}n\H{o} m\'odos\'{\i}t\'as a ki\'{\i}\'o met\'odusban van, 
a legels\H{o} t\'enyleges argumentumk\'ent (param\'eterk\'ent) belevett\"uk a forr\'asf\'ajl 
nev\'et. A t\'enyleges h\'{\i}v\'askor a {\tt \_\_line\_\_} v\'altoz\'ot haszn\'altuk, amelyet a 
Python automatikusan mindenkor be\'all\'{\i}t az \'eppen v\'egreajtott \'allom\'any nev\'ere. Ez 
abba az esetben is igaz, ha import\'alter modulr\'ol van sz\'o. 

Most folytassuk a kor\'abbi lista alapj\'an az oszt\'alyunk b\H{o}v\'{\i}t\'es\'et, pr\'ob\'aljuk 
meg beilleszteni a bejegyz\'es id\H{o}pontj\'at. Ez az\'ert annyira nem lesz egyszer\H{u}, kell 
hozz\'a egy \'ujabb modul. A Python r\'esz\'et k\'epez\H{o} {\tt time} modulr\'ol van sz\'o, 
amelyet az oszt\'aly defin\'{\i}ci\'o el\H{o}tt import\'alni kell:

\begin{Verbatim}[fontsize=\small]
## tesztelni

## tesztelni
## 2. plussz, idopont


import time

class Naplo:

    .....

    def Kiir(self, forras, txt):
        timeStmp = time.strftime("%Y-%m-%d:%H:%M:%S",time.localtime())
        msg = '%s > %s > %s\n'%(forras,timeStmp,txt)
        self.puffer.append(msg)
        self.sor += 1
        if self.sor == self.meret:
            if self.start:
                trc = open(self.logNev,self.startMode)
            else:
                trc = open(self.logNev,self.mode)
            for idx in xrange(self.meret):
                trc.write(self.puffer[idx])
            trc.close()
            self.start = False
            self.puffer = []
            self.sor = 0

\end{Verbatim}

\noindent A fenti k\'odban ism\'et csak a ki\'{\i}r\'o met\'odus m\'odosult, \'es ehhez egy 
kis magyar\'azatot szeretn\'enk hozz\'af\H{u}zni. A Python futtat\'o rendszer le tudja k\'erdezni 
a sz\'am\'{\i}t\'og\'ep bels\H{o} \'or\'aj\'at, erre valo a {\tt time.localtime()} f\"uggv\'eny. 
Ennek eredm\'enye egy bels\H{o} adatszerkezetet, amelyet ut\'ana sz\"ovegg\'e kell alak\'{\i}tani 
a megfelel\H{o}en felparam\'etezett {\tt time.strftime()} h\'{\i}v\'assal. Az els\H{o} argumentum 
az id\H{o}pont kimeneti form\'atum\'at adja meg "{\it \'ev-h\'onap-nap:\'ora:perc:m\'asodperc}" 
alakban, a m\'asodik pedig maga a pontos id\H{o}t szolg\'altat\'o h\'iv\'as. (A seg\'edv\'altoz\'ok 
elker\"ul\'es\'enek ezt a m\'odj\'at nagyon sok Python programoz\'o kedveli.)

Ahogyan a k\'odb\'oll\'atszik, a napl\'obejegyz\'es form\'atum\'an is v\'altoztattunk egy kiss\'e, 
nem puszt\'an annyit, hogy beraktuk az id\H{o}pontot. A h\'arom fontos r\'eszt elv\'alasztottuk 
egym\'ast\'ol egy jellel, hogy jobban olvashat\'o legyen. Emellett a m\'odos\'{\i}t\'as azt is 
lehet\H{o}v\'e teszi, hogy a logot ak\'ar egy m\'asik programmal is fel tudjuk dolgozni. \'Erdemes 
kipr\'ob\'alni ezt a m\'odos\'{\i}t\'ast is, v\'elem\'eny\"unk szerint az eddigiek alapj\'an az 
olvas\'oink is meg tudj\'ak tenni. A kiirat\'o met\'odus param\'eterei ugyanis nem v\'altoztak.

A t\'emak\"or bevezet\H{o} gondolatai k\"oz\"ott megeml\'{\i}tett\"uk, hogy m\'as-m\'as jelleg\H{o} 
adatok kell a napl\'oba ki\'{\i}rni a tesztel\'esi \'es a haszn\'ati szakaszban. Az ut\'obbi esetben 
csak a hib\'akat akarjuk feljegyezni. Ez lesz az utols\'o b\H{o}v\'{\i}t\'es, amelyet az oszt\'alyunkba 
bevezet\"unk:

\begin{Verbatim}[fontsize=\small]
## tesztelni

class Naplo2:

    def __init__(self, pNeve, pMeret, pMode = 'a', pSzint = 2 ):
        self.logNev = pNeve
        self.meret = pMeret
        self.startMode = pMode
        self.mode = 'a'
        self.sor = 0
        self.puffer = []
        self.start = True
        self.szint = pSzint
        self.tipus = ['INFO', 'HIBA']

    def __del__(self):
        if self.sor != 0:
            if self.start:
                trc = open(self.logNev,self.startMode)
            else:
                trc = open(self.logNev,self.mode)
            for idx in xrange(self.meret):
                trc.write(self.puffer[idx])
            trc.close()

    def szintAllit(self, pUjSzint):
        self.szint = pUjSzint

    def Kiir(self, forras, txt, pSzint = 2):
        if pSzint >= self.szint:
            timeStmp = time.strftime("%Y-%m-%d:%H:%M:%S",time.localtime())
            szintNev = self.tipus[pSzint - 1]
            msg = '%s > %s > %s > %s\n'%(szintnev forras,timeStmp,txt)
            self.sor += 1
            if self.sor == self.meret:
                if self.start:
                    trc = open(self.logNev,self.startMode)
                else:
                    trc = open(self.logNev,self.mode)
                for idx in xrange(self.meret):
                    trc.write(self.puffer[idx])
                trc.close()
                self.start = False
                self.puffer = []
                self.sor = 0

\end{Verbatim}

\noindent Ez a m\'odos\'{\i}t\'as most azonban nem volt olyan egyszer\H{u}, mint a kor\'abbiak, 
viszont ez\'altal m\'ar t\'enyleg haszn\'alhat\'o oszt\'alyt kaptunk. A legfontosabb szempont, ami 
vezetett minket a r\'ezletek kital\'al\'asakor, hogy a tesztelt k\'odhoz k\'epest az "\'eles" program 
lehet\H{o}leg csak a legsz\"uks\'egesebb m\'odon v\'altozzon. Ehhez els\H{o}sorban \'es el\H{o}sz\"or 
a napl\'obejegyz\'es form\'atum\'at v\'altoztattuk meg, bevezett\"unk egy t\'{\i}pust mind egyes 
log sorba. Ezek \'ert\'eke n\"ovekv\H{o} "s\'ulyoss\'agi sorrendben" {\tt INFO}, illetve {\tt HIBA}. 
Ez alapj\'a be\'all\'{\i}that\'o, hogy milyen szint\H{u} \"uzenetek ker\"uljenek bele a napl\'oba. 
Erre szolg\'al {\tt szintAllit} met\'odus. Ha a ki\'{\i}r\'o met\'odusban megadott szint nem nagyobb, 
vagy egyenl\H{o} a be\'all\'{\i}tottal, akkor be sem ker\"ul a logba. Ilyen m\'odon a fejleszt\H{o}i 
k\'odban elegend\H{o} a napl\'oz\'asi szintet m\'odos\'{\i}tani, m\'aris k\'esz az "\'eles" rendszer.
Javasoljuk az olvas\'oinknak, hogy ezt az oszt\'alyt is pr\'ob\'aljuk ki, megszerzett tud\#asuk ehhez 
b\H{o}s\'egesen elegend\H{o}. Nem mellesleg, ezzel m\'ar egy t\'enylegesen is j\'ol haszn\'alhat\'o 
logol\'o oszt\'alyt is siker\"ult k\"oz\"sen megalkotni. Lehetne m\'eg az oszt\'alyunkat ezzel-azzal 
b\H{o}v\'{\i}teni, (p\'eld\'aul tov\'abbi napl\'oz\'asi szintekkel), de ezt m\'ar nem kiv\'anjuk 
kidolgozni. A lehet\H{o}s\'eg mindenki el\H{o}tt nyitva \'all.

\subsection{A Python szabv\'anyos napl\'oz\'o modulja}
\index{logger}

Az el\H{o}z\H{o}ekben \'{\i}rtunk napl\'oz\'o f\"uggv\'enyt, \'es egyszer\H{u}bb esetekben igen j\'ol 
haszn\'alhat\'o logol\'o oszt\'alyt. Mind a k\'et esetben megjegyezt\"uk, hogy a tov\'abbi b\H{o}v\'{\i}t\'es 
lehets\'eges ugyan, de nem gyakorlatias, mert a Python programoz\'asi nyelv rendelkezik be\'ep\'{\i}tett, 
szabv\'anyos erre a c\'elra szolg\'al\'o modullal. Ezt a {\tt logging} modul tartalmazza, de az \'altalunk 
meg\'{\i}rtakn\'al l\'enyegesen sokoldal\'ubb \'es rugalmasabb. Ebb\H{o} is k\"ovetkezik, hogy fel\'ep\'{\i}t\'ese 
t\"obb r\'etegb\H{o}l \'all:

\begin{itemize}
   \item{{\tt -} form\'az\'o, amelyik a bejegyz\'est kin\'ezet\'et meghat\'arozza,}
   \item{{\tt -} eloszt\'o, amelyik a bejegyz\'est a megfelel\H{o} napl\'oba eljuttatja,}
   \item{{\tt -} sz\H{u}r\H{o}, meghat\'arozza, hogy milyen s\'ulyoss\'ag\'u \"uzenetek ker\"uljenek a 
      napl\'oba,}
   \item{{\tt -} a napl\'o k\"ozvetlen\"ul haszn\'aland\'o szintje.}      
\end{itemize} 

\noindent Ez alapj\'an mindenki azt gondolhatja, hogy nagyon neh\'ez haszn\'alni, de szerencs\'ere ez 
az\'ert nincsn \'{\i}gy. Csak megjegyezz\"uk, hogy az eloszt\'o szint t\'enyleg nagyon sokat tud, a 
bejegy'eseket ak\'ar email-ben is el tudja k\"uldeni, vagy egyszer\H{u}en egy t\'avoli g\'epre is ki 
tudja \'{\i}rni. Mi csak egyszer\H{u}en \'allom\'anyba fogjuk jegyezni, ahogyan a kor\'abbiakban is. 
Kezdj\"unk is akkor bele:

%% TODO formazs??

\begin{Verbatim}[fontsize=\small]
import logging 

logName = 'proba.log'
    
handler = logging.FileHandler(logName, 'a')    
formatter = logging.Formatter("%(name)s %(asctime)s %(filename)s %(lineno)d %(levelname)s %(message)s")
handler.setFormatter(formatter)

\end{Verbatim}

\noindent Annak \'erdek\'eben, hogy a fenti r\'eszletet pontosan meg\'erts\"uk, soronk\'ent v\'egigmegy\"unk 
a k\'odon. Az \'allom\'any nev\'enek be\'all\'{\i}t\'asa ut\'an l\'etrehozzuk a f\'ajlkezel\H{o}t (FileHandler), 
\'es a megnyit\'asi m\'odot hhozz\'af\H{u}z\'esre \'all\'{\i}tjjuk be. A m\'asodik sorban a form\'at adjuk 
meg az\'altal, hogy defini\'aljuk az form\'az\'o objektumot. Ezt a form\'ay\'ot csatoljuk a f\'ajlkeze\H{o}h\"oz. 
A bejegyz\'es most el\'egg\'e r\'eszletes inform\'aci\'ot fog tartalmazni:

\begin{itemize}
   \item{{\tt -} a napl\'o csatorna neve, amelyet majd k\'es\H{o}bb fogunk megadni (name),}
   \item{{\tt -} a bejegyz\'es pontos ideje (ascitime),}
   \item{{\tt -} a Python f\'ajl neve (filename),}
   \item{{\tt -} a h\'{\i}v\'as hely\'enek sorsz\'ama a f\'ajlban (lineno),}
   \item{{\tt -} a bejegyz\'es "\'ulyoss\'agi" szintje (level),}
   \item{{\tt -} maga a bejegyz\'es (message)}
\end{itemize} 

\noindent A k\'odban a z\'ar\'ojelek ut\'an \'allnak a m\'ar ismert kimenetet form\'az\'o karakterek. 
N\'ezz\"uk tov\'abb:

\begin{Verbatim}[fontsize=\small]
logger = logging.getLogger('Proba')
logger.addHandler(handler)

logger.setLevel(logging.DEBUG)
\end{Verbatim}

\noindent A {\tt getLogger} h\'{\i}v\'assal el\H{o}\'all\'{\i}tjuk mag\'at a napl\'o csatorn\'at (napl\'o), 
majd ehhez hozz\'aadjauk a f\'ajlkezel\H{o}t (handler). Az import\'alt modul a {\tt logger} objektumban a 
rendelkez\'es\"unkre bocs\'ajtja a {\tt setLevel} met\'odust, amellyel megsz\H{u}rherj\"uk, hogy milyen 
\"uzenetek menjenek ki a napl\'oba. A {\tt logging.DEBUG} \'ert\'ek minden szintet enged\'elyez. 

\begin{tabular}{lll}
   {\bf N\'ev} & {\bf \'Ert\'ek} & {\bf Magyar\'ayat}\\
   {\tt logging.NOTSET} & $0$ & Az \'ert\'ek nem lett be\'all\'{\i}tva\\
   {\tt logging.DEBUG} & $10$ & R\'eszleetes nyomk\"ovet\'es\\
   {\tt logging.INFO} & $20$ & Fontos inform\'aci\'o\\
   {\tt logging.WARNING} & $30$ & Figyelmeztet\'es, a m\H{u}k\"od\'est zavar\'o esem\'eny\\
   {\tt logging.ERROR} & $40$ & Hiba\\
   {\tt logging.CRITICAL} & $40$ & S\'ulyos, kritikus hiba\\
\end {tabular}

\noindent Ahogyan a fenti t\'abl\'azatb\'ol l\'atszik, hogy minn\'el s\'ulyosabb egy hiba, illetve 
fontosabb egy bejegyz\'es, ann\'al nagyobb sz\'amszer\H{u} \'ert\'ek van hozz\'arendelve. Ez a 
sz\H{u}r\H{o} m\'odszer alapja. Most m\'ar ideje, hogy ennyi el\H{o}k\'esz\'{\i}t\'es ut\'an l\'assunk 
egy t\'enyleges logol\'ast:

\begin{Verbatim}[fontsize=\small]
logger.info('Program indul')
\end{Verbatim}

\noindent Az \'{\i}gy \"ossze\'all\'atott program lefuttat\'asa alapj\'an a napl\'oban az al\'abbi 
bejegyz\'es tal\'aljuk:

\begin{Verbatim}[fontsize=\small]
Proba 2020-09-27 11:57:06,629 proba.py 14 INFO Program indul
\end{Verbatim}

\noindent L\'atjuk hogy az el\H{o}z\H{o}kben r\'eszletesen le\'{\i}rt form\'atumnak teljesen megfelel 
a logba be\'{\i}rt sor. Most n\'ezz\"uk meg, hogyan is m\H{u}k\"odik a sz\H{u}r\H{o}:

\begin{Verbatim}[fontsize=\small]
logger.setLevel(logging.WARNING)

logger.info('Program indul')
\end{Verbatim}

\noindent Ha ezzel a m\'odos\'{\i}t\'assal lefuttatjak a szkriptet, akkor a napl\'oban mem fog 
megjelenni semmi, mert a be\'all\'{\i}tott szint sz\'amszer\H{u} \'ert\'eke magasabb, mint a 
logol\'o utas\'{\i}t\'as \'altal be\'all\'{\i}tott sz\'anszer\H{u} \'ertek. Nem meglep\H{o}, 
hogy a t\"obbi szintet ebben a p\'eld\'aban a {\tt logger.debug()}, {\tt logger.warning()}, 
{\tt logger.error()}, illetve a {\tt logger.critical()} met\'odusokkal irathatjuk bele a logba. 

A {\tt logging} modul sok rendelkez\'es\"unkre \'all\'o lehet\H{o}s\'ege k\"oz\"ul m\'eg csak egyet 
fogunk itt megmutatni, mert a v\'elem\'eny\"unk szerint nagyon hasznos a mindennapi gyakorlatban. 
Ha bonyolultabb programot \'{\i}runk, amely esetleg t\"obb modulb\'ol \'all, akkor term\'eszetesen 
mer\"ul fel az ig\'eny, hogy a k\"ul\"onb\"oz\H{o} alkot\'oelemek napl\'obejegyz\'esei ne keveredjenek 
\"oosze. Az els\H{o} \"otlet ami ezzek kapcsolatban az esz\"unkbe juthat, hogy a bejegyz\'eseket az 
egyes programr\'eszek k\"ul\"on \'allom\'anyba \'{\i}rj\'ak bele. Ez eddig nagyon egyszer\H{u}nek 
t\H{u}nik, de ha ut\'olag meg kell n\'ezni, hogy mis is t\"ort\'ent a futtat\'askor, akkor el\'egg\'e 
macer\'as feladat lenne, hogy egyn\'el t\"obb logb\'ol kellene a t\"ort\'enetet \"osszebogar\'aszni. 
Az ide\'alis az lenne, hogy egyh helyre menn\'enek a bejegyz\'esek, de valahogyan m\'egis meg tudn\'ank 
a komponenst k\"ul\"onb\"oztetni. Szerencs\'ere erre van m\'odszer, amelyet al\'abb mutatunk (a t\"obbi 
k\'odr\'eszletet a kor\'abbiak alapj\'an az olvas\'ink is \"ossze tudj\'ak szedni): 

\begin{Verbatim}[fontsize=\small]

logger = logging.getLogger('Proba')
logger.addHandler(handler)

testlogger = logging.getLogger('Proba.test')

logger.setLevel(logging.DEBUG)

logger.info('Program indul')
testlogger.debug('Test start')
\end{Verbatim}

\noindent A kimenet:

\begin{Verbatim}[fontsize=\small]
Proba 2020-09-28 09:22:00,585 proba_a.py 16 INFO Program indul
Proba.test 2020-09-28 09:22:00,605 proba_a.py 17 DEBUG Test start
\end{Verbatim}

\noindent A napl\'oba azonos form\'atummal, de k\'et k\"ul\"un logol\'o bejegyz\'esei ker\"ultek, a 
k\'et "csatorna" teh\'egy \"osszef\"ugg. Ebben az esetben a f\H{o} naplo a {\tt Proba}, \'es ennek 
mintegy lesz\'armazottja a {\tt Proba.test}. A k\"ul\"onb\"oz\"o napl\'ok sz\H{u}r\'esi felt\'etelei 
teljesen megegyeznek. 

\input py2_cont

\input py2_app


